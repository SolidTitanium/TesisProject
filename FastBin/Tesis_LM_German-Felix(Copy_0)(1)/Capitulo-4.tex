\chapter{Homolog\'ia Persistente}

La homolog\'ia persistente es una herramienta poderosa que es usada para el computo, estudio y codificaci\'on
multiescala de propiedades topol\'ogicas de familias anidadas de complejos simpliciales y espacios
topol\'ogicos. No solo provee algoritmos eficientes para calcular los n\'umeros de Betti de cada complejo
en las familias consideradas, como se requiere para la inferencia homol\'ogica cubierta en la secci\'on
anterior, sino que tambi\'en codifica la evoluci\'on de los grupos de homolog\'ia de los
complejos anidados a trav\'es de las escalas. Ideas y resultados preliminares que culminan en la
teor\'ia de la homolog\'ia persistente pueden ser encontrados desde antes del siglo XXI, en particular
en los trabajos de Barannikov (1994) \cite{Barannikov1994}, Frosini (1992) \cite{Frosini1992},
Robins (1999) \cite{Robins1999}; pero su desarrollo en su forma moderna se concreto en los trabajos
de Edelsbrunner et al. (2002) \cite{Edelsbrunner2002} y Zomorodian y Carlsson (2005)
\cite{Zomorodian2005}.

\section{Filtraciones}

Una filtraci\'on de un complejo simplicial $K$ es una familia anidada de subcomplejos
$\cpar{K_{r}}_{r\in t}$, donde $T\subseteq\mathbb{R}$, tal que para cualquier
$r, r' \in T$, si $r\leq r'$ entonces $K_{r}\subseteq K_{r'}$ y $K = \bigcup_{r\in T}K_{r}$.
El subconjunto $T$ puede ser finito o infinito. En general, una filtraci\'on de un espacio
topol\'ogico $\mathbb{M}$ es una familia anidada de subespacios $\cpar{M_{r}}_{r\in T}$,
donde $T\subseteq\mathbb{R}$, tal que para cualquier $r, r' \in T$,
si $r\leq r'$ entonces $M_{r}\subseteq M_{r'}$ y $M = \bigcup_{r\in T}M_{r}$. Por ejemplo, si
$f: \mathbb{M}\rightarrow\mathbb{R}$ es una funci\'on, entonces la familia
$M_{r} = f^{-1}\cpar{\left(-\infty, r\right]}$, $r\in\mathbb{R}$ define una filtraci\'on llamada la
filtraci\'on del conjunto subnivel de $f$.

En la pr\'actica, el par\'ametro $r\in T$ suele ser interpretado como un par\'ametro de escala, y las
filtraciones com\'unmente usadas en el ATD suelen pertenecer a uno de los siguientes dos tipos.

\section*{Filtraciones Sobre Datos}

Dado un subconjunto $\mathbb{X}$ de un espacio m\'etrico compacto $\cpar{M,\rho}$, las familias de
complejos de Vietoris-Rips $\cpar{Rips_{r}\cpar{\mathbb{X}}}_{r\in\mathbb{R}}$ y los complejos de \v Cech
$\cpar{Cech_{r}\cpar{\mathbb{X}}}_{r\in\mathbb{R}}$ son filtraciones\footnote{Aqu\'i consideramos
$Rips_{r}\cpar{\mathbb{X}} = Cech_{r}\cpar{\mathbb{X}} = \varnothing$, si $r<0$}. Aquí, el par\'ametro
$r$ puede ser interpretado como la resoluci\'on con la que se considera el conjunto de datos $\mathbb{X}$.
Por ejemplo, si $\mathbb{X}$, es una nube de puntos en $\mathbb{R}^{d}$, gracias al teorema del nervio,
la filtraci\'on $\cpar{Cech_{r}\cpar{\mathbb{X}}}_{r\in\mathbb{X}}$ codifica la topolog\'ia de todo la
familia de uniones de bolas $\mathbb{X}^{r} = \cup_{x\in\mathbb{X}}B\cpar{x,r}$, cuando $0<r<\infty$.
Como la noci\'on de filtraci\'on es algo flexible, se han considerado muchas otras filtraciones en la
literatura para ser construidas sobre los datos, como el complejo testigo popularizado en el ATD por
De Silva y Carlsson (2004)\cite{DeSilva2004}, las filtraciones de Rips con peso Buchet et al.
(2015)\cite{Buchet2015b}, o las filtraciones DTM Anai et al. (2019)\cite{Anai2019} que nos permiten
trabajar con conjuntos de datos con ruido o con datos at\'ipicos.

\begin{figure}[ht]
    \centering
    \includegraphics[width=0.85\linewidth]{./figures/Figura9.png}
    \caption{
        El c\'odigo de barras de persistencia y el diagrama de persistencia de la funci\'on
        $f:\ccorch{0,1}\rightarrow\mathbb{R}$.
    }
    \label{fig:Figura 9}
    \vspace{15pt}
\end{figure}

\section*{Filtraciones de Conjuntos Subnivel}

Definir funciones en los v\'ertices de un complejo simplicial da lugar a otro importante ejemplo de
filtraci\'on: sea $K$ el complejo simplicial con el conjunto de v\'ertices $V$ y
$f:V\rightarrow\mathbb{R}$. Entonces $f$ puede ser extendida a todos los simplices de $K$ definiendo
$f\cpar{\ccorch{v_{0},\dots,v_{k}}} = \max\cllav{f\cpar{v_{i}}: i=1,\dots,k}$ para cualquier simplejo
$\sigma = \ccorch{v_{0},\dots,v_{k}}\in K$ y la familia de subcomplejos,
$K_{r}=\cllav{\sigma\in K:f\cpar{\sigma}\leq r}$, define una filtraci\'on llamada la filtraci'on del
conjunto subnivel de $f$. La filtraci\'on del conjunto sobre-nivel de $f$ se define de manera similar.

En la pr\'actica, incluso si el \'indice del conjunto es infinito, todas las filtraciones consideradas
son construidas en conjuntos finitos y son, en si, finitas. Por ejemplo, cuando $\mathbb{X}$ es finito,
el complejo de Vietoris-Rips $Rips_{r}\cpar{\mathbb{X}}$ cambia solo en un numero finito de \'indices,
$r$. Esto nos permite manejarlos de manera sencilla desde una perspectiva algebraica.

\section{Algunos Ejemplos}

Dada una filtraci\'on $\mathit{Filt} = \cpar{F_{r}}_{r\in T}$ de un complejo simplicial o un espacio
topol\'ogico, la homolog\'ia de $F_{r}$ cambia cuando $r$ incrementa; pueden aparecen nuevos componentes
conexos y algunos ya existentes pueden unirse, aros y cavidades pueden formarse o llenarse, etc. La
homolog\'ia persistente registra estos cambios, identifica las propiedades que aparecen y asocia un
tiempo de vida a cada una. La informaci\'on resultante se codifica como un conjunto de intervalos llamado
c\'odigo de barras, o bien, como un conjunto de puntos en $\mathbb{R}^{2}$ donde la coordenada de
cada punto es el punto de inicio y final de cada intervalo correspondiente.

Antes de dar una definici\'on formal, ilustraremos el concepto de homolog\'ia persistente con unos
ejemplos.

% IDE Change, change to hard-wrapping per sentence instead of hard-wrapping per max number of char.

\section*{Ejemplo 1}

Sea $f:\ccorch{0,1}\rightarrow\mathbb{R}$ la funci\'on de la Figura \ref{fig:Figura 9},
y sea $F_{r} = f^{-1}\cpar{\cpar{-\infty,r}}_{r\in\mathbb{R}}$ la filtraci\'on del conjunto subnivel de $f$.
Todos los conjuntos subnivel de $f$ son o bien vacios o la uni\'on de intervalos,
as\'i que la \'unica informaci\'on topol\'ogica no-trivial que brindan es su homolog\'ia cero dimensional, esto es, su n\'umero de componenetes conexas.
Para $r<a_{1}$, $F_{r}$ es vacio, pero para $r = a_{1}$, aparecen un primer componente conexo en $F_{a_{1}}$.
La homolog\'ia persistente registra $a_{1}$ como la ``fecha de nacimiento'' de una componente conexa la codifica como un intervalo que comienza en $a_{1}$.
Luego, $F_{r}$ permanece conexo hasta que $r$ toma el valor de $a_{2}$ donde una segunda componente conexa aparece.