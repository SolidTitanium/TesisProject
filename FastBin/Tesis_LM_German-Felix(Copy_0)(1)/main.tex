\documentclass[letterpaper,twoside,11pt]{book}
\usepackage{amsmath,amsfonts,amssymb,amsthm,mathrsfs}
\usepackage[spanish]{babel}
\usepackage{makeidx,multicol}
\usepackage[colorlinks,citecolor={black},linkcolor={black}]{hyperref}
\usepackage{graphicx}
\usepackage{MnSymbol}
\usepackage{fancyhdr}
\usepackage{float}
\usepackage[all,cmtip]{xy}
\usepackage{tikz}
\usetikzlibrary{positioning}

%Self
\usepackage{enumitem}

\usepackage{multicol}
\usepackage{mathtools}

\usepackage[latin1]{inputenc}
% \usepackage[sort&compress]{natbib}
% \usepackage{makeidx}

\input xy
\xyoption{all}

\theoremstyle{definition}
\newtheorem{teorema}{Teorema}[chapter]
\newtheorem{conjetura}[teorema]{Conjectura}
\newtheorem{corolario}[teorema]{Corolario}
\newtheorem{definicion}[teorema]{Definici\'on}
\newtheorem{lema}[teorema]{Lema}
\newtheorem{proposicion}[teorema]{Proposici\'on}
\newtheorem{observacion}[teorema]{Observaci\'on.}
\newtheorem{nota}[teorema]{Nota}
\newtheorem{ejem}[teorema]{Ejemplo}

%Comandos
\newcommand{\cpar}[1]{\left(#1\right)}
\newcommand{\ccorch}[1]{\left[#1\right]}
\newcommand{\cllav}[1]{\left\{#1\right\}}
\newcommand{\cabs}[1]{\left|#1\right|}
\newcommand{\cnorm}[1]{\left\|#1\right\|}
%Fin Comandos

\newenvironment{demostracion}[1][Demostraci\'on]{\noindent\nobreak\textbf{#1.} }{\ \rule{0.6em}{0.6em}}

%This change labels of subfig
% \renewcommand{\thesubfigure}{\alph{subfigure}\arabic{subfiggroup}}
% \captionsetup[subfigure]{labelformat=simple,labelsep=colon,
%                          listofformat=subsimple}
%\captionsetup{lofdepth=2} This is in order to list the subfigures in the LOF
% \makeatletter
%  \renewcommand{\p@subfigure}{}
%   %Esto lo agrego yo para tener subfiguras a1, b1, ... a2, b2, ... 
%   %Se reinicia cada vez que una nueva figura es convocada (como es debido).
%   \newcounter{subfiggroup}[figure] 
% \makeatother

%\usepackage{epsfig}
%\usepackage{url}
%Esto genera enlaces en el PDF


%Este mejora las prestaciones de "\verbatim"
\usepackage{verbatim}

%Definicion de margenes

%\usepackage[left=2cm,top=2.8cm,right=2cm,bottom=2.5cm]{geometry}
\usepackage[left=4.5cm,top=2.8cm,right=2cm,bottom=2.5cm]{geometry}  %%%%% ORIGINAL %%%%%
\sloppy
\pagestyle{empty}

% Code for creating empty pages
% No headers on empty pages before new chapter
\makeatletter
\def\cleardoublepage{\clearpage\if@twoside \ifodd\c@page\else
    \hbox{}
    \thispagestyle{plain}
    \newpage
    \if@twocolumn\hbox{}\newpage\fi\fi\fi}
\makeatother \clearpage{\pagestyle{plain}\cleardoublepage}

% Code for creating fully-empty pages
% Fully empty pages before command is called
\makeatletter
\def\clearfullypage{\clearpage\if@twoside \ifodd\c@page\else
    \hbox{}
    \thispagestyle{empty}
    \newpage
    \if@twocolumn\hbox{}\newpage\fi\fi\fi}
\makeatother \clearpage{\pagestyle{empty}\clearfullypage}


%\includeonly{edicion}
%\includeonly{jurado,apoyo,licencia,edicion,reconocimientos,tvs}

%Print subsubsection numbers and put them in TOC
\setcounter{secnumdepth}{3}
\setcounter{tocdepth}{1}
\raggedbottom

\begin{document}


%%%%%%%%%%%%%
\frontmatter
%%%%%%%%%%%%%


\pagestyle{empty}
|	\hspace*{-37mm}
\begin{tabular}{p{3cm}p{15.0cm}}
\includegraphics[width=2.9cm]{logo_unison.png}

\begin{center}
\rule[2cm]{1.5mm}{16.5cm}%vertical
\hspace{2pt}
\rule[2cm]{0.7mm}{16.5cm}%vertical
\hspace{2pt}
\rule[2cm]{1.5mm}{16.5cm}%vertical
\end{center}

&
\vspace{-3.4cm}
\begin{center}
{\LARGE{ \bf{UNIVERSIDAD DE SONORA}}}
\\
\rule[0mm]{15.0cm}{0.2mm}%horizontal
\\
\rule[3mm]{15.0cm}{1.2mm}%horizontal
\\
\Large{DIVISI\'ON DE CIENCIAS EXACTAS Y NATURALES}

\vspace{0.8\baselineskip}

\Large{\bf Programa de Licenciatura en Matem\'aticas}

\vspace{2.8\baselineskip}

{\Large \bf{Titulo de la tesis}}


\vspace*{0.7cm}

\LARGE{\bf T\ E\ S\ I\ S}

\vspace*{4mm}

{\Large Que para obtener el t\'itulo de:}

\vspace*{4mm}

{\Large \bf Licenciado en Matem\'aticas}

\vspace*{4mm}

{\Large Presenta:}

\vspace*{4mm}

{\Large Nombre del tesista}

\vspace*{1.0cm}


{\Large Director de tesis: Prof. Jes\'us Francisco Espinoza Fierro}

\vspace*{1.2cm}
\small{Hermosillo, Sonora, M\'exico}\hspace*{4cm}\small{... de 20xx}

\end{center}

\end{tabular}

\newpage

\thispagestyle{empty}

\clearfullypage
\pagenumbering{roman}
\chapter*{Sinodales}

\noindent \textbf{...}\\
...,\\
...
\bigskip

\noindent \textbf{Dr. Fulano de tal}\\
Departamento de Matem\'aticas,\\
Universidad de Sonora
\bigskip


\noindent \textbf{Sutano...}\\
Departamento de Matem\'aticas,\\
Universidad de Sonora
\bigskip


\noindent \textbf{Mangano...}\\
Departamento de Matem\'aticas,\\
Universidad de Sonora
\bigskip


\clearfullypage
\newpage
\vspace*{8cm} 

\begin{center}
...

\end{center}
\clearfullypage
\include{Agradecimientos}

\clearfullypage
\pagestyle{fancy}
\fancyhf{}
\renewcommand{\chaptermark}[1]{\markboth{ \textbf{#1}}{}}
\fancyhead[LO]{}
\fancyhead[LO]{}
\fancyfoot[LE,RO]{\thepage}

% Redefine plain page style
\fancypagestyle{plain}{
\fancyhf{}
\renewcommand{\headrulewidth}{0pt}
\fancyfoot[LE,RO]{\thepage}
}
 
% Dutch style of paragraph formatting, i.e. no indents.
\setlength{\parskip}{1.3ex plus 0.2ex minus 0.2ex}

% Remove parskip for toc
\setlength{\parskip}{0ex plus 0.5ex minus 0.2ex}

\tableofcontents

\cleardoublepage

% Adjustments headers
\fancyhead[RO]{\leftmark}
\addcontentsline{toc}{chapter}{Introducci\'on}
\chapter*{Introducci\'on}
%\fancyhf{}
%\rhead{INTRODUCCI\'ON}

El an\'alisis topol\'ogico de datos (ATD) es un campo reciente 
que emerge de varios trabajos en topolog\'ia (algebraica) aplicada
y la geometr\'ia computacional durante la primera d\'ecada del siglo \textbf{XXI}. Aunque
es posible encontrarse con acercamientos geom\'etricos al
an\'alisis de datos desde mucho antes, el ATD comenz\'o a desarrollarse
como un campo con los trabajos de Edelsbrunner et al. (2002) \cite{Edelsbrunner2002} y
Zomorodian y Carlsson (2005) \cite{Zomorodian2005} en homolog\'ia persistente, el campo fue
popularizado en un destacado art\'iculo en 2009 \cite{Carlsson2005}.
El ATD es motivado principalmente por la idea que la topolog\'ia y la
geometr\'ia brindan un acercamiento poderoso para inferir de manera robusta
caracter\'isticas cualitativas y cuantitativas sobre la estructura
de un conjunto de datos [e.g., Chazal (2017) \cite{Chazal2017}].

El objetivo del ATD es generar m\'etodos matem\'aticos, estad\'isticos y algor\'itmicos
bien fundamentados para inferir, analizar y explotar las complejas estructuras topol\'ogicas
y geom\'etricas subyacentes a datos que usualmente son representados como nubes de puntos en
espacios Euclideanos o espacios m\'etricos m\'as generales. En el transcurso de los \'ultimos
a\~{n}os se ha realizado un esfuerzo considerable para proporcionar estructuras de datos robustas
y eficientes, adem\'as de algoritmos para ATD que actualmente son implementados y
facilitados a trav\'es de paqueter\'ias est\'andar como la paqueter\'ia
GUDHI\footnote{https://gudhi.inria.fr/} (C++ y Python) Maria et al. (2014) \cite{Maria2014} y su
interfaz para el software R, Fasy et al. (2014a) \cite{Fasy2014a},
Dionysus\footnote{https://www.mrzv.org/software/dionysus/},
PHAT\footnote{https://bitbucket.org/phat-code/phat},
DIPHA\footnote{https://github.com/DIPHA/dipha} o
Giotto\footnote{https://giotto-ai.github.io/gtda-docs/0.4.0/library.html}.
Aunque evoluciona con rapidez, el ATD proporciona un conjunto de herramientas maduras y
eficientes que pueden ser usadas de manera complementaria o conjunta a otras herramientas
de la ciencia de datos.

\section*{Estructura General del An\'alisis Topol\'ogico de Datos}

El ATD se ha desarrollado recientemente en m\'ultiples direcciones y campos de aplicaci\'on.
Actualmente existe una variedad de m\'etodos inspirados por acercamientos topol\'ogicos y
geom\'etricos. Dar un resumen que cubra con entereza de los acercamientos
existentes se encuentra fuera del alcance de esta introducci\'on. Sin embargo, muchos m\'etodos
est\'andar siguen la siguiente secuencia:

\begin{enumerate}
    \item Suponemos que la entrada de datos es un conjunto finito de puntos con una noci\'on
    de distancia o similitud entre ellos. Esta puede ser inducida por una m\'etrica en el
    espacio de entrada (e.g. la m\'etrica Euclidiana si se trata de datos inmersos en
    $\mathbb{R}^{d}$) o ser una m\'etrica intr\'inseca definida por una matriz de distancia
    por pares. La definici\'on de la m\'etrica en los datos normalmente es parte de la entrada
    o es guiada por la aplicaci\'on. No obstante, es importante notar que la elecci\'on de
    dicha m\'etrica puede ser cr\'itica para revelar caracter\'isticas topol\'ogicas y
    geom\'etricas interesantes de los datos.
    
    \item Se construye una figura ``continua'' sobre los datos con el prop\'osito de resaltar
    las estructuras topol\'ogicas y geom\'etricas subyacentes. Usualmente se trata de
    un complejo simplicial o una familia anidada de complejos simpliciales, llamada filtraci\'on,
    la cual refleja la estructura de los datos en diferentes escalas. Los complejos simpliciales
    pueden ser vistos como generalizaciones de gr\'aficas vecinales que cl\'asicamente son
    construidas sobre los datos en muchos tipos de an\'alisis o algoritmos de aprendizaje.
    El desaf\'io aqu\'i es definir tales estructuras de tal manera que sean capaces de
    reflejar informaci\'on relevante acerca de la estructura de los datos y que puedan ser
    construidas de manera efectiva y manipuladas en la pr\'actica.
    
    \item Informaci\'on topol\'ogica y geom\'etrica es extra\'ida de las estructuras construidas
    sobre los datos. Esto puede resultar en una reconstrucci\'on completa, t\'ipicamente una
    triangulaci\'on, de la forma subyacente de los datos de los cuales se pueden extraer
    f\'acilmente propiedades topol\'ogicas y geom\'etricas en forma de res\'umenes o
    aproximaciones las cuales requieren m\'etodos espec\'ificos, como la homolog\'ia persistente,
    para la extracci\'on de informaci\'on relevante. M\'as all\'a de la identificaci\'on
    de informaci\'on topol\'ogica/geom\'etrica interesante y su visualizaci\'on e
    interpretaci\'on, el desaf\'io en este paso es mostrar su relevancia, en particular su
    estabilidad con respecto a las perturbaciones o la presencia de ruido en los datos de entrada.
    Es por ello que entender el comportamiento estad\'istico de las propiedades inferidas es
    tambi\'en una cuesti\'on importante.
    
    \item La informaci\'on topol\'ogica y geom\'etrica proporciona una nueva familia de
    caracter\'isticos y descriptores de los datos. Estos pueden ser usados para entender mejor
    los datos (en particular a trav\'es de visualizaci\'on) o pueden ser combinados con otros
    tipos de caracter\'isticos para un an\'alisis posterior o tareas de aprendizaje autom\'atico.
    Esta informaci\'on tambi\'en puede ser utilizada para dise\~{n}ar modelos bien ajustados para el
    an\'alisis de datos o el aprendizaje autom\'atico. Mostrar el valor a\~{n}adido y complementario
    (con respecto a otras caracter\'isticas) de la informaci\'on proporcionada por las
    herramientas del ATD es un punto importante en este paso.
    
\end{enumerate}

\section*{El An\'alisis Topol\'ogico de Datos y la Estad\'istica}

Hasta hace poco, los aspectos te\'oricos del TDA y la inferencia topol\'ogica reca\'ian
principalmente en acercamientos determin\'isticos. Estos acercamientos no tomaban en cuenta la
naturaleza aleatoria de los datos y la variabilidad intr\'inseca de las cantidades topol\'ogicas
que infieren. As\'i, la mayor\'ia de los m\'etodos correspondientes son de car\'acter explicativo,
sin ser capaces de distinguir eficientemente entre informaci\'on y lo que normalmente es llamado
``ruido topol\'ogico''.

Un acercamiento estad\'istico al ATD implica considerar los datos como generados de una
distribuci\'on desconocida y a su vez que las propiedades topol\'ogicas inferidas utilizando
m\'etodos del ATD son vistos como estimadores de cantidades topol\'ogicas que describen un
objeto subyacente. Bajo este acercamiento, el objeto desconocido usualmente corresponde al soporte
de la distribuci\'on de los datos (o parte del mismo). Los objetivos principales de un acercamiento
estad\'istico al an\'alisis topol\'ogico de datos pueden ser abreviados como la siguiente lista
de problemas:

\begin{enumerate}[label=\emph{T\'opico \arabic*:}]
    \item Demostrar consistencia y estudiar la tasa de convergencia de los m\'etodos del ATD.
    
    \item Proporcionar regiones de confianza para caracter\'isticas topol\'ogicas y discutir la
    significacia de las cantidades topol\'ogicas estimadas.
    
    \item Seleccionar escalas relevantes en las cuales el fen\'omeno topol\'ogico debe ser considerado,
    en funci\'on de los datos observados.
    
    \item Lidiar con valores at\'ipicos y brindar m\'etodos robustos para el ATD.
    
\end{enumerate}

\section*{Aplicaciones del An\'alisis Topol\'ogico de Datos en la Ciencia de Datos.}

Desde el punto de vista de las aplicaciones, recientemente hay muchos resultados prometedores que han
demostrado la eficacia de acercamientos topol\'ogicos y geom\'etricos en una multitud de campos, tales
como la ciencia de materiales (Kramar et al., 2013 \cite{Kramar2013}; Nakamura
et al., 2015 \cite{Nakamura2015}; Pike et al., 2020 \cite{Pike2020}), an\'alisis de
formas 3D (Skraba et al., 2010 \cite{Skraba2010}; Turner et al., 2014b \cite{Turner2014b}), an\'alisis
de im\'agenes (Qaiser et al., 2019 \cite{Qaiser2019}; Rieck et al., 2020 \cite{Rieck2020}), an\'alisis
de series de tiempo multivariadas (Khasawneh y Munch, 2016 \cite{Khasawneh2016};
Seversky et al., 2016 \cite{Seversky2016}; Umeda, 2017 \cite{Umeda2017}),
medicina (Dindin et al., 2020 \cite{Dindin2020}), biolog\'ia (Yao et al., 2009 \cite{Yao2009}),
gen\'omica (Carri\`ere y Rabad\'an, 2020 \cite{Carriere2020}), qu\'imica (Lee et al.,
\cite{Lee2017}; Smith et al., 2021 \cite{Smith2021}), redes sensoriales (De Silva y Ghrist, 2007
\cite{Silva2007}) y transportaci\'on (Li et al., 2019 \cite{Li2019}), entre otros. Dar una lista
exhaustiva de las aplicaciones del ATD esta fuera del alcance de esta introducci\'on. Por otra parte,
la mayor\'ia de los resultados del ATD son fruto de su combinaci\'on con otras t\'ecnicas de an\'alisis
y aprendizaje. De esta manera vemos que clarificar la posici\'on y complementariedad del ATD con respecto
a otros acercamientos y herramientas en la ciencia de datos es una cuesti\'on importante y un campo de
investigaci\'on activo.

As\'i, los objetivos generales de este documento son los siguientes. Primero, se intenta proporcionar
a los analistas de datos con una breve pero exhaustiva introducci\'on a los fundamentos matem\'aticos
y estad\'isticos del ATD. Con este prop\'osito, nos enfocamos en una selecci\'on de herramientas
y t\'opicos, los complejos simplicales y su uso para el an\'alisis topol\'ogico de datos exploratorio,
la inferencia geom\'etrica  y la homolog\'ia persistente, los cuales juegan un rol central en el ATD.
Segundo, se apunta a demostrar como, gracias al reciente progreso del software, herramientas del ATD
pueden ser f\'acilmente aplicadas en la ciencia de datos. En particular, mostraremos como la versi\'on
de Python de la paqueter\'ia GUDHI permite una sencilla implementaci\'on y uso de las herramientas
presentadas. Nuestro objetivo es proporcionar al analista de datos referencias relevantes de manera que
se obtenga un comprensi\'on clara de los elementos b\'asicos del ATD y sea capaz de utilizar sus
m\'etodos y software en un conjunto propio de problemas y datos.

Otros estudios del ATD, complementarios a este trabajo, pueden ser encontrados en la literatura.
Wasserman (2018) \cite{Wasserman2018} presenta una perspectiva estad\'istica al ATD, y se concentra, en
particular, en las conexiones entre el ATD y el agrupamiento por densidad. Sizemore et al. (2019)
\cite{Sizemore2019} propuso un estudio acerca de las aplicaciones del ATD a las neurociencias.
Finalmente, Hensel et al. (2021) \cite{Hensel2021} presenta un resumen de las aplicaciones
del ATD al aprendizaje autom\'atico.

%%%%%%%%%%%%%
\mainmatter
%%%%%%%%%%%%%



\pagenumbering{arabig} 
\pagenumbering{arabic}
% Adjustments headers
\fancyhead[RO]{\leftmark}
\fancyhead[EL]{\textbf{Cap\'itulo \thechapter}}
\setcounter{page}{1}


%%%%%%%%%%%%%%%%%%%%%%%%%%%%%%%%%%%%%%%%%%%%%%%%%%%%%%%%%%%%%%%%%%%%%%%%%%%%%%%%%%%%%%
%%%%%%%%%%%%%%%%%%%%%%%%%%%%%%%%%%%%%%%%%%%%%%%%%%%%%%%%%%%%%%%%%%%%%%%%%%%%%%%%%%%%%%
%%%%%%%%%%%%%%%%%%%%%%%%%%%%%%%%%%%%%%%%%%%%%%%%%%%%%%%%%%%%%%%%%%%%%%%%%%%%%%%%%%%%%%
%%%%%%%%%%%%%%%%%%%%%%%%%%%   CONTENIDO DE LA TESIS   %%%%%%%%%%%%%%%%%%%%%%%%%%%%%%%%
%%%%%%%%%%%%%%%%%%%%%%%%%%%%%%%%%%%%%%%%%%%%%%%%%%%%%%%%%%%%%%%%%%%%%%%%%%%%%%%%%%%%%%
%%%%%%%%%%%%%%%%%%%%%%%%%%%%%%%%%%%%%%%%%%%%%%%%%%%%%%%%%%%%%%%%%%%%%%%%%%%%%%%%%%%%%%
%%%%%%%%%%%%%%%%%%%%%%%%%%%%%%%%%%%%%%%%%%%%%%%%%%%%%%%%%%%%%%%%%%%%%%%%%%%%%%%%%%%%%%

\chapter{Espacios M\'etricos, Coberturas y Complejos Simpliciales}

Debido a que las caracter\'isticas topol\'ogicas y geom\'etricas suelen ser asociadas con espacios continuos,
datos representados como una conjunto finito de observaciones no revelan informaci\'on topol\'ogica
directamente. Una manera natural de revelar alg\'un tipo de estructura topol\'ogica en los datos
es ``conectar'' puntos de datos que se encuentren cerca con el prop\'osito de exhibir una forma continua
global subyacente en los datos. Usualmente cuantificamos la noci\'on de cercania entre puntos utilizando
una distancia (o medida de disimilaridad), y muchas veces resulta conveniente considerar conjuntos de datos
como espacios m\'etricos discretos o muestras de espacios m\'etricos. Esta secci\'on introduce conceptos
generales para la inferencia geom\'etrica y topol\'ogica; una presentaci\'on m\'as completa del tema se
encuentra en el estudio por Boissonnat et al. (2018) \cite{Boissonnat2018}.

\section*{Espacios M\'etricos}

Recordemos que un espacio m\'etrico $\left(M, \rho\right)$ es un conjunto $M$ con una funci\'on
$\rho: M \times M \rightarrow \mathbb{R}_{+}$, llamada distancia, tal que para cualquier $x, y, z \in M$,
se tiene lo siguiente:

\begin{enumerate}[label=\roman*)]
    \item $\rho\left(x, y\right) \geq 0$ y $\rho\left(x, y\right) = 0$ si y s\'olo si $x = y$,
    
    \item $\rho\left(x, y\right) = \rho\left(y, x\right)$, y
    
    \item $\rho\left(x, z\right) \leq \rho\left(x, y\right) + \rho\left(y, z\right)$.
    
\end{enumerate}

Dado un espacio m\'etrico $\left(M, \rho\right)$, el conjunto de subconjuntos compactos de
$\left(M, \rho\right)$ denotado por $\mathcal{K}\left(M\right)$, puede ser dotado
con la distancia de Hausdorff; dados dos subconjuntos compactos $A, B \subseteq M$,
la distancia de Hausdorff $d_{H}\left(A, B\right)$ entre $A$ y $B$ es
definida como el n\'umero no negativo m\'as peque\~{n}o $\delta$, tal que para cualquier $a \in A$, existe
$b \in B$ de manera que $\rho\left(a, b\right) \leq \delta$ (Figura \ref{fig:Figura 1}). En otras palabras,
si dado cualquier subconjunto compacto $C \subseteq M$, denotamos por $d\left(\cdot, C\right):
M\rightarrow\mathbb{R}_{+}$ a la funci\'on distancia de $C$ definida por
$d\left(x, C\right) \coloneqq \inf_{c\in C}\rho\left(x, c\right)$ para cualquier $x \in M$, entonces se
puede probar que la distancia de Hausdorff entre $A$ y $B$ esta definida por una de las siguientes
igualdades:

\begin{align*}
    d_{H}\left(A, B\right) & = \max\left\{\sup_{b\in B}d\left(b, A\right),
    \sup_{a\in A}d\left(a, B\right)\right\} \\
    & = \sup_{x\in M}\left|d\left(x, A\right) - d\left(x, B\right)\right| =
    \left\|d\left(\cdot, A\right) - d\left(\cdot, B\right)\right\|_{\infty}
\end{align*}

Es un resultado cl\'asico que la distancia de Hausdorff es en efecto una distancia en el conjunto de
subconjuntos compactos de un espacio m\'etrico. Desde la perspectiva del ATD, esta distancia
brinda una manera conveniente de cuantificar la proximidad entre diferentes conjuntos de datos que
provienen del mismo espacio m\'etrico. Sin embargo, a veces es necesario comparar conjuntos de datos que
no son muestreados del mismo espacio. Por fortuna la noci\'on de la distancia de Hausdorff puede ser
generalizada para comparar cualquier par de espacios m\'etricos compactos, esta es la idea de la distancia
de Gromov-Hausdorff.

Dados dos espacios m\'etricos compactos, $\left(M_{1}, \rho_{1}\right)$ y $\left(M_{2}, \rho_{2}\right)$,
decimos que son isom\'etricos si existe una biyecci\'on $\phi: M_{1}\rightarrow M_{2}$ que preserva
distancias, esto es, $\rho_{2}\left(\phi\left(x\right), \phi\left(y\right)\right) =
\rho_{1}\left(x, y\right)$ para cualquier $x, y \in M_{1}$. La distancia de Gromov-Hausdorff mide cuan lejos
est\'an dos espacios m\'etricos de ser isom\'etricos.

\begin{definicion}
    La distancia de Gromov-Hausdorff $d_{GH}\left(M_{1}, M_{2}\right)$ entre dos espacios m\'etricos
    compactos es el \'infimo de los n\'umeros reales $r \geq 0$ tal que existe un espacio m\'etrico
    $\left(M, \rho\right)$ y dos subespacios compactos $C_{1}$ y $C_{2} \subset M$ que son isom\'etricos
    a $M_{1}$ y $M_{2}$ y que cumplen $d_{H}\left(C_{1}, C_{2}\right) \leq r$.
\end{definicion}

\begin{figure}[h]
    \centering
    \includegraphics[width=0.85\linewidth]{./figures/Figura1_c1.png}
    \caption{
        Izquierda: la distancia de Hausdorff entre dos subconjuntos $A$ y $B$ en el plano. en este ejemplo,
        $d_{H}\left(A, B\right)$ es la distancia entre el punto $a$ en $A$ que es el m\'as lejano a $B$ y
        su vecino m\'as cercano $b$ en $B$. Derecha: la distancia de Gromov-Hausdorff entre $A$ y $B$. $A$
        puede ser rotado para reducir su distancia de Hausdorff a $B$. As\'i, $d_{GH}\left(A, B\right) \leq
        d_{H}\left(A, B\right)$.
    }
    \label{fig:Figura 1}
    \vspace{15pt}
\end{figure}

Usaremos la distancia de Gromov-Hausdorff m\'as adelante para el estudio de las propiedades de estabilidad
de los diagramas de persistencia.

Conectar pares de puntos de datos cercanos mediante aristas lleva a la noci\'on est\'andar de una gr\'afica
simple de la cual la conectividad de los datos puede ser analizada usando, por ejemplo, algoritmos de
agrupamiento. Para ir m\'as all\'a de la conectividad, una idea central en el ATD es construir nociones
equivalentes a las gr\'aficas simples pero de dimensi\'on m\'as alta, utilizando no s\'olo pares sino
$\left(k + 1\right)$-tuplas de puntos de datos cercanos. El resultado son objetos llamados complejos
simpliciales, los cuales nos ayudan a identificar nuevas caracter\'isticas topol\'ogicas tales como ciclos,
huecos, y sus correspondientes de dimensiones superiores.

\section*{Complejos Simpliciales Geom\'etricos y Abstractos}

Los complejos simplicales pueden considerarse como gr\'aficas generalizadas a dimensiones superiores.
Son objetos matem\'aticos que son de naturaleza topol\'ogica y combinatoria a la vez, una propiedad que
los hace particularmente \'utiles para el ATD.

Dado un conjunto $\mathbb{X} = \left\{x_{0}, \dots, x_{k}\right\}\subset\mathbb{R}^{d}$ con $k + 1$
puntos af\'inmente independientes, el simplejo $k$-dimensional
$\sigma = \left[x_{0}, \dots, x_{k}\right]$ generado por $\mathbb{X}$ es la envolvente convexa de
$\mathbb{X}$. Los puntos de $\mathbb{X}$ son llamados v\'ertices de $\sigma$, y los simplejos generados
por los subconjuntos de $\mathbb{X}$ son llamados caras de $\sigma$. Un complejo simplicial geom\'etrico
$K$ en $\mathbb{R}^{d}$ es una colecci\'on de simplejos que cumplen lo siguiente:

\begin{enumerate}[label=\roman*)]
    \item Cualquier cara de un simplejo de $K$ es un simplejo de $K$ y,
    
    \item La intersecci\'on de cualesquiera dos simplejos de $K$ es el conjunto vac\'io o una cara com\'un
    de ambos simplejos.
    
\end{enumerate}

La uni\'on de los simplejos de $K$ es un subconjunto de $\mathbb{R}^{d}$ llamado el espacio subyacente
de $K$ que hereda la topolog\'ia de $\mathbb{R}^{d}$. As\'i, $K$ puede ser visto como un espacio
topol\'ogico a trav\'es de su espacio subyacente. Es de notar que una vez que se conocen los v\'ertices,
$K$ se encuentra completamente caracterizado por la descripci\'on combinatoria de una colecci\'on de
simplejos que satisfacen ciertas reglas de incidencia.

Dado un conjunto $V$, un complejo simplicial abstracto con un conjunto de v\'ertices $V$ es un conjunto
$\tilde{K}$, de subconjuntos finitos de $V$ tales que los elementos de $V$ pertenecen a $\tilde{K}$ y
que para cualquier $\sigma \in \tilde{K}$, cualquier subconjunto de $\sigma$ pertenece a $\tilde{K}$.
Los elementos de $\tilde{K}$ son llamados las caras o los simplejos de $\tilde{K}$. La dimensi\'on
de un simplejo abstracto es su cardinalidad menos 1 y la dimensi\'on de $\tilde{K}$ es la mayor de las
dimensiones de sus simplejos. Es de notar que los complejos simpliciales de dimensi\'on $1$
son gr\'aficas.

La descripci\'on combinatoria de cualquier complejo simplicial geom\'etrico $K$ da lugar a un complejo
simplicial abstracto $\tilde{K}$. El inverso tambi\'en es cierto; siempre es posible asociar con un
complejo simplical abstracto $\tilde{K}$ un cierto espacio topol\'ogico $|\tilde{K}|$
tal que si $K$ es un complejo simplicial geom\'etrico cuya descripci\'on combinatoria es la misma
que la de $\tilde{K}$, entonces el espacio subyacente de $K$ es homeomorfo a $|\tilde{K}|$.
Dicha $K$ es llamada una realizaci\'on geom\'etrica de $\tilde{K}$. Como consecuencia de esto,
los complejos simpliciales abstractos pueden ser vistos como espacios topol\'ogicos y los complejos
simpliciales geom\'etricos pueden ser vistos como realizaciones geom\'etricas de la estructura combinatoria
subyacente. As\'i, se puede considerar a los complejos simpliciales como objetos combinatorios que se
ajustan bien a c\'alculos computacionales efectivos y a su vez como espacios topol\'ogicos de los cuales se pueden
inferir propiedades topol\'ogicas.

\section*{Construcci\'on de Complejos Simpliciales a partir de Datos}

Dado un conjunto de datos, o m\'as generalmente, un espacio m\'etrico o topol\'ogico, existen varias
maneras de construir complejos simpliciales. Esta es una presentaci\'on de algunos ejemplos cl\'asicos
que son usados con frecuencia en la pr\'actica.

Comenzando con una extensi\'on inmediata de la noci\'on de una gr\'afica, Sup\'ongase que
tenemos un conjunto de puntos $\mathbb{X}$ en un espacio m\'etrico $\left(M, \rho\right)$ y un n\'umero
real $\alpha \geq 0$. El complejo de Vietoris-Rips $Rips_{\alpha}\left(\mathbb{X}\right)$ es el conjunto
de simplejos $\left[x_{0}, \dots, x_{k}\right]$ tal que
$\rho_{\mathbb{X}}\left(x_{i}, x_{j}\right)\leq\alpha$ para todo
$\left(i, j\right)$, ver Figura \ref{fig:Figura 2}. De aqu\'i vemos que el complejo de
Vietoris-Rips es efectivamente
un complejo simplicial abstracto. Aunque, en general, incluso cuando $\mathbb{X}$ es un subconjunto
finito de $\mathbb{R}^{d}$, $Rips_{a}\left(\mathbb{X}\right)$ no admite una realizaci\'on geom\'etrica en
$\mathbb{R}^{d}$; en particular, puede ser de una dimensi\'on mayor a $d$, por ejemplo, si se tienen $d+2$
puntos en $R^{d}$ que cumplen $\rho_{\mathbb{X}}\left(x_{i},x_{j}\right)\leq\alpha$ para todo
$\left(i,j\right)$, entonces $Rips_{a}\left(\mathbb{X}\right)$ es de dimensi\'on $d+1$, podemos ver un
caso similar en el tetraedro formado en el complejo derecho de la Figura \ref{fig:Figura 2}.

Estrechamente relacionado al complejo de Vietoris-Rips est\'a el complejo de
\v Cech $Cech_{a}\left(\mathbb{X}\right)$ el cual se define como el conjunto de simplejos
$\left[x_{0}, \dots, x_{k}\right]$ tales que las $k + 1$ bolas cerradas $B\left(x_{i},\alpha\right)$
tienen intersecci\'on no vac\'ia, ver Figura \ref{fig:Figura 2}. Estos dos complejos estan relacionados por

\begin{equation*}
    Rips_{\alpha}\left(\mathbb{X}\right)\subseteq
    Cech_{\alpha}\left(\mathbb{X}\right)\subseteq
    Rips_{2\alpha}\left(\mathbb{X}\right)
\end{equation*}

\noindent y que si $\mathbb{X}\subset\mathbb{R}^{d}$, entonces $Cech_{\alpha}\left(\mathbb{X}\right)$ y
$Rips_{2\alpha}\left(\mathbb{X}\right)$ tienen el mismo esqueleto $1$-dimensional, esto es, comparten el
mismo conjunto de v\'ertices y aristas.

\begin{figure}[ht]
    \centering
    \includegraphics[width=0.85\linewidth]{./figures/Figura2_c1.png}
    \caption{
        El complejo de \v Cech, $Cech_{\alpha}\left(\mathbb{X}\right)$ (izquierda) y el de Vietoris-Rips
        $Rips_{2\alpha}\left(\mathbb{X}\right)$ (derecha) en una nube finita de puntos en
        $\mathbb{R}^{2}$. La parte inferior de $Cech_{\alpha}\left(\mathbb{X}\right)$ es la uni\'on de
        dos tri\'angulos adyacentes, mientras que la parte inferior de
        $Rips_{2\alpha}\left(\mathbb{X}\right)$ es el tetraedro generado por los cuatro v\'ertices
        y todas sus caras. La dimensi\'on del complejo de \v Cech es $2$. La dimensi\'on del complejo
        de Vietoris-Rips es $3$. Es de notar que el complejo de Vietoris-Rips, en este caso, no puede ser inmerso en $\mathbb{R}^{2}$.
    }
    \label{fig:Figura 2}
    \vspace{15pt}
\end{figure}

\section*{El Teorema del Nervio}

El complejo de \v Cech es un caso particular de una familia de complejos asociados con cubiertas. Dada
una cubierta $\mathcal{U}=\left(U_{i}\right)_{i\in I}$ de $\mathbb{M}$, conjunto de puntos en
$\mathbb{R}^{d}$, es decir, una familia de conjuntos $U_{i}$ tales que
$\mathbb{M}=\cup_{i\in I}U_{i}$, el nervio de $\mathcal{U}$ es el complejo simplicial
abstracto $C\left(\mathcal{U}\right)$ cuyos v\'ertices son los $U_{i}$'s y que cumple

\begin{equation*}
    \sigma = \left[U_{i_{0}}, \dots, U_{i_{k}}\right] \in C\left(\mathcal{U}\right)
    \text{ si y s\'olo si } \cap_{j=0}^{k}U_{i_{j}}\neq\varnothing.
\end{equation*}

Dada una cubierta de un conjunto de datos, donde cada conjunto de la cubierta es, por ejemplo, una
agrupaci\'on de los puntos de los datos que tienen ciertas propiedades en com\'un,
su nervio proporciona una descripci\'on combinatoria, compacta y global, de las relaciones
entre estos conjuntos a trav\'es de sus patrones de intersecci\'on. Ver Figura (\ref{fig:Figura 3}).

\begin{figure}[ht]
    \centering
    \includegraphics[width=0.85\linewidth]{./figures/Figura3.png}
    \caption{
        Nube de puntos muestreada en el plano y una cubierta de conjuntos abiertos para esta
        nube (izquierda). El nervio de esta cubierta es un tri\'agulo (derecha).
        Los v\'ertices corresponden a uno de los conjuntos de la cubierta mientras que
        las aristas corresponden a una de las
        intersecciones no vac\'ias entre dos conjuntos de la cubierta.
    }
    \label{fig:Figura 3}
    \vspace{15pt}
\end{figure}

Un teorema fundamental en topolog\'ia algebraica se encarga de relacionar, bajo ciertas condiciones, la
topolog\'ia del nervio de una cubierta con la topolog\'ia de la uni\'on de los conjuntos
de dicha cubierta. Es necesario introducir algunas nociones adicionales para ser
formales a la hora de enunciar este
resultado conocido como el teorema del nervio.

Dos espacios topol\'ogicos, $X$ y $Y$, usualmente son considerados iguales desde un punto de vista
topol\'ogico si son homeomorfos, esto es, si existen dos funciones, biyectivas y continuas,
$f:X\rightarrow Y$ y $g:Y\rightarrow X$ tales que $f\circ g$ y $g\circ f$ son las funciones
identidad de $Y$ y $X$, respectivamente. En muchas ocasiones, pedir que $X$ y $Y$ sean
homeomorfos resulta ser una condici\'on demasiado fuerte para asegurar que $X$ y $Y$
compartan propiedades topol\'ogicas de inter\'es para el ATD.
Dos funciones continuas $f_{0}, f_{1}:X\rightarrow Y$ se dicen ser homot\'opicas
si existe una funci\'on continua $H:X\times\left[0, 1\right]\rightarrow Y$ tal que para
cualquier $x\in X$, $H\left(x, 0\right) = f_{0}\left(x\right)$ y $H\left(x, 1\right) = g\left(x\right)$.
Los espacios $X$ y $Y$ se dicen ser homot\'opicamente equivalentes si existen
dos funciones, $f:X\rightarrow Y$ y $g:Y\rightarrow X$, tales que $f\circ g$ y $g\circ f$ son
homot\'opicas a la func\'on identidad de $Y$ y $X$, respectivamente.
Las funciones $f$ y $g$ son llamadas homot\'opicamente equivalentes. La noci\'on de
equivalencia homot\'opica es m\'as d\'ebil que la de homeomorfismo; si $X$ y $Y$ son homeomorfos, entonces
son homot\'opicamente equivalentes, pero el rec\'iproco no es cierto. Sin embargo, espacios que son
homot\'opicamente equivalentes a\'un comparten muchos invariantes topol\'ogicos, como la conexidad por
caminos, los grupos de homotop\'ia y, en particular, tienen la misma homolog\'ia.

Un espacio se dice ser contra\'ible si es homot\'opicamente equivalente a un punto. Las bolas, y en
general los conjuntos convexos en $\mathbb{R}^{d}$, son ejemplos b\'asicos de espacios contra\'ibles.
Las cubiertas abiertas, para las cuales se tiene que todos sus elementos e intersecciones son
contra\'ibles, tienen la siguiente propiedad.

\begin{teorema}[Teorema del Nervio]\label{teoNervio}
    Sea $\mathcal{U} = \left(U_{i}\right)_{i\in I}$ una
    cubierta abierta de un espacio topol\'ogico $X$ tal que la intesecci\'on
    de cualquier subcolecci\'on de los $U_{i}$'s es contra\'ible o vac\'ia.
    Entonces, $X$ y el nervio $C\left(\mathcal{U}\right)$ son
    homot\'opicamente equivalentes.
\end{teorema}

Es f\'acil verificar que subconjuntos convexos de espacios euclidianos son contra\'ibles. Como
consecuencia, si $\mathcal{U} = \left(U_{i}\right)_{i\in I}$ es una colecci\'on de subconjuntos convexos
de $\mathbb{R}^{d}$, entonces $C\left(\mathcal{U}\right)$ y $\cup_{i\in I}U_{i}$ son homot\'opicamente
equivalentes. En particular, si $\mathbb{X}$ es un conjunto de puntos en $\mathbb{R}^{d}$, entonces el
complejo de \v Cech $Cech_{\alpha}\left(\mathbb{X}\right)$ es homot\'opicamente equivalente a la uni\'on
de bolas $\cup_{x\in\mathbb{X}}B\left(x, \alpha\right)$.

El teorema del nervio juega un papel fundamental en el ATD; proporciona una manera de codificar la
topolog\'ia de espacios continuos en estructuras combinatorias abstractas que se ajustan con facilidad
al dise\~{n}o de estructuras de datos y algoritmos efectivos.
\chapter{Utilizando Cubiertas y Nervios para el An\'alisis de Datos Exploratorio y Visualizaci\'on: El
Algoritmo Mapper.}

Usar el nervio de cubiertas como una manera de visualizar y explorar datos es una idea natural que fue
propuesta para el ATD en el estudio por Singh et al. \cite{Singh2007}, dando lugar al algoritmo Mapper.

\begin{definicion}
    Sea $f:\mathbb{X}\rightarrow\mathbb{R}^{d}$, $d\geq1$, una funci\'on continua y sea
    $\mathcal{U} = \left(U_{i}\right)_{i\in I}$ una cubierta de $\mathbb{R}^{d}$. laL cubierta pull-back
    de $\mathbb{X}$ inducida por $\left(f, \mathcal{U}\right)$ es la colecci\'on de conjuntos abiertos
    $\left(f^{-1}\left(U_{i}\right)\right)_{i\in I}$. El pull-back refinado es una colecci\'on de
    componentes conexas de los abiertos $f^{-1}\left(U_{i}\right)$, $i\in I$.
\end{definicion}

La idea del algoritmo Mapper es, dado un conjunto de datos $\mathbb{X}$ y una funci\'on
$f:\mathbb{X}\rightarrow\mathbb{R}^{d}$, sintetizar $\mathbb{X}$ a trav\'es del nervio
del pull-back refinado de una cubierta $\mathcal{U}$ de $f\left(\mathbb{X}\right)$. Para cubiertas bien
escogidas $\mathcal{U}$, este nervio es una gr\'afica que encapsula de manera conveniente el detalle de los
datos y los vuelve f\'aciles de visualizar (Ver Figura \ref{fig:Figura 4}).

El algoritmo de Mapper es muy sencillo; pero este recalca las diferentes elecciones que son dejadas al
usuario y que discutiremos a continuaci\'on.

\begin{itemize}
    \item \textbf{Entrada:} Un conjunto de datos $\mathbb{X}$ con una m\'etrica o medida de disimilaridad
    entre los puntos asociados a los datos, una funci\'on $f:\mathbb{X}\rightarrow\mathbb{R}$
    (o bien, $f:\mathbb{X}\rightarrow\mathbb{R}^{d}$), y una cubierta $\mathcal{U}$
    de $f\left(\mathbb{X}\right)$.
    Para cada $U\in\mathcal{U}$, descomponer $f^{-1}\left(U\right)$ en agrupaciones
    $C_{U,1}, \dots, C_{U,k_{U}}$. Calcular el nervio de la cubierta de $X$ definido por los
    $C_{U,1}, \dots, C_{U,k_{U}}$, $U\in\mathcal{U}$.
    
    \item \textbf{Salida:} Un complejo simplicial; el nervio que incluye un v\'ertice $v_{U,i}$ por cada
    $C_{U,i}$ y una arista entre cada uno de los v\'ertices $v_{U,i}$ y $v_{U',j}$ que cumplan
    $C_{U,i}\cap C_{U',j} \neq \varnothing$.
    
\end{itemize}

\newpage

\begin{figure}[ht]
    \centering
    \includegraphics[width=0.85\linewidth]{./figures/Figura4.png}
    \caption{
        (A) Cubierta pull-back refinada de la funci\'on altura sobre una superficie en $\mathbb{R}^{3}$.
        (B) Algoritmo de Mapper en una nube de puntos muestreada alrededor de un c\'irculo y la
        funci\'on altura.
    }
    \label{fig:Figura 4}
    \vspace{15pt}
\end{figure}

\section*{La Elecci\'on de $f$}

La elecci\'on de la funci\'on $f$, a veces llamada la funci\'on filtro o lente, depende fuertemente de las
propiedades de los datos que uno pretende resaltar. Las siguientes son algunas de las m\'as encontradas en
la literatura:

\begin{itemize}
    \item Estimadores de densidad: El complejo Mapper puede ser \'util para entender la estructura y
    conexidad de \'areas de alta densidad.
    
    \item Coordenadas de an\'alisis de componentes principales (coordenadas PCA) o funciones coordenadas
    obtenidas de una t\'ecnica de reducci\'on de dimensionalidad no lineal (NLDR), eigenfunciones de
    laplacianos de gr\'aficas pueden ayudar a revelar y entender parte de la ambig\"uedad en el uso
    de reducciones de dimensionalidad no lineales.
    
    \item La funci\'on de centralidad $f\left(x\right) = \Sigma_{y\in\mathbb{X}}d\left(x,y\right)$ y la
    funci\'on de excentricidad $f\left(x\right) = \max_{y\in\mathbb{X}}d\left(x,y\right)$ a veces resultan
    ser buenas elecciones que no requieren de ning\'un conocimiento espec\'ifico acerca de los datos.
    
    \item Para datos muestreados sobre estructuras filamentarias de dimensi\'on uno, la funci\'on
    distancia a un punto dado permite recuperar la topolog\'ia subyacente de las estructuras filamentarias
    \cite{Chazal2015d}.
    
\end{itemize}

\section*{La Elecci\'on de la Cubierta $\mathcal{U}$}

Cuando $f$ es una funci\'on de valores reales, una elecci\'on est\'andar de $\mathcal{U}$
es un conjunto de intervalos espaciados regularmente y del mismo largo, $r>0$, cubriendo al conjunto
$f\left(\mathbb{X}\right)$. El n\'umero real $r$ es a veces llamado la resoluci\'on de
la cubierta, y el porcentaje $g$ de sobreposici\'on entre dos
intervalos consecutivos es llamado la ganancia de la cubierta.
N\'otese que si la ganancia $g$ es escogida menor a $50\%$, entonces cada punto de la linea real es
cubierto por, a lo m\'as, $2$ conjuntos abiertos de $\mathcal{U}$, y el nervio
resultante es una gr\'afica. Es importante notar que la salida de Mapper es muy
sensible a la elecci\'on de $\mathcal{U}$, y cambios
peque\~{n}os en la resoluci\'on o ganancia puede afectar de manera significativa al resultado, volviendo el
m\'etodo muy inestable. Una estrategia cl\'asica consiste en explorar un rango de par\'ametros y
seleccionar aquellos que sean m\'as informativos desde el punto de vista del usuario.

\section*{La Elecci\'on del Agrupamiento}

El algoritmo Mapper requiere el agrupamiento de la preimagen de conjuntos abiertos $U\in\mathcal{U}$.
Existen dos estrategias para realizar este agrupamiento. La primera consiste en aplicar, a cada
$U\in\mathcal{U}$, un algoritmo de agrupamiento, escogido por el usuario, a la preimagen de
$f^{-1}\left(U\right)$. La segunda, m\'as global, consiste en construir una gr\'afica sobre el
conjunto de datos $\mathbb{X}$, por ejemplo, una gr\'afica k-NN o una $\epsilon$-gr\'afica y, para cada
$U\in\mathcal{U}$, tomar las componentes conexas de la subgr\'afica con el conjunto de v\'ertices
$f^{-1}\left(U\right)$.

\section*{Aspectos Teor\'eticos y Estad\'isticos del Algoritmo Mapper}

Basados en los resultados de estabilidad y la estructura de Mapper propuestos en el estudio por
Carri\`ere y Oudot (2017) \cite{Carriere2017}, se han realizado avances en direcci\'on a
una versi\'on de Mapper estad\'isticamente bien fundamentada en el estudio por Carri\`ere et al. (2018)
\cite{Carriere2018}. De aqu\'i destaca que la convergencia de Mapper depende tanto del muestreo de los
datos como de la regularidad de la funci\'on filtro. M\'as aun, estrategias de submuestreo pueden ser
usadas para seleccionar un complejo en una filtraci\'on de Rips a una escala conveniente, as\'i como la
resoluci\'on y la ganancia para definir la gr\'afica Mapper. El caso para filtros estoc\'asticos y
multivariados tambi\'en ha sido estudiado por Carri\`ere y Michel (2019) \cite{Carriere2019}.
Una descripci\'on alternativa de la convergencia probabil\'istica de Mapper, en t\'erminos de la
categorificaci\'on, fue propuesta en el estudio por Brown et al. (2020) \cite{Brown2020}. Otros
acercamientos tambi\'en fueron propuestos para estudiar y lidiar con la inestabilidad del algoritmo
Mapper en los trabajos de Dey et al. (2016) \cite{Dey2016}, Dey et al. (2017) \cite{Dey2017}.

\section*{An\'alisis de Datos con Mapper}

Como una herramienta del an\'alisis de datos, Mapper se ha utilizado con \'exito para tareas de
agrupamiento y selecci\'on de atributos. La idea es identificar estructuras
espec\'ificas en la gr\'afica (o complejo) Mapper, en particular, lazos.
Estas estructuras son usadas para identificar
c\'umulos interesantes o seleccionar atributos que puedan diferenciar los datos en estas estructuras de
manera apropiada. Aplicaciones en datos reales ilustrando estas t\'ecnicas pueden ser encontradas en,
por ejemplo, los estudios por Carri\`ere y Rabad\'an (2020) \cite{Carriere2020}, Lum et al. (2013)
\cite{Lum2013}, Yao et al. (2009) \cite{Yao2009}.
\chapter{Reconstrucci\'on Geom\'etrica e Inferencia Homol\'ogica}

Otra forma de construir cubiertas y usar sus nervios para exhibir la estructura topol\'ogica de los datos
es considerar la uni\'on de bolas centradas en los puntos de los datos. En esta secci\'on suponemos que
$\mathbb{X}_n = \left\{x_0, \dots, x_n\right\}$ es un subconjunto de $\mathbb{R}^{d}$, muestrado de manera
i. i. d. de acuerdo con la medida de probabilidad $\mu$ con soporte compacto $M\subset\mathbb{R}^{d}$. La
estrategia general para inferir informaci\'on topol\'ogica acerca de $M$ a trav\'es de $\mu$ consiste en
dos pasos:

\begin{enumerate}
    \item Se cubre $\mathbb{X}_{n}$ con una uni\'on de bolas de radio fijo con centros en las $x_{i}$'s.
    Bajo algunas condiciones de regularidad en $M$, se puede relacionar la topolog\'ia de esta uni\'on de
    bolas con la de $M$.
    
    \item Desde un una perspectiva pr\'actica y algor\'itmica, las cualidades topol\'ogicas de $M$ son
    inferidas del nervio de la uni\'on de las bolas, utilizando el teorema del nervio.
    
\end{enumerate}

De esta manera, es posible comparar espacios a trav\'es de equivalencias isot\'opicas, una noci\'on m\'as
fuerte que la de homeomorfismo: $X\subseteq\mathbb{X}^{d}$ y $Y\subseteq\mathbb{X}^{d}$ se dicen ser
(ambientalmente) isot\'opicos si existe una familia continua de homeomorfismos
$H: \left[0, 1\right]\times\mathbb{R}^{d}\rightarrow\mathbb{R}^{d}$, $H$ continua, tal que, para cualquier
$t\in\left[0, 1\right]$, $H_{t} = H\left(t, \cdot\right):\mathbb{R}^{d}\rightarrow\mathbb{R}^{d}$ es un
homeomorfismo, $H_{0}$ es el mapeo identidad en $\mathbb{R}^{d}$, y $H_{1}\left(X\right)=Y$.
Es claro que, si $X$ y $Y$ son isot\'opicos, entonces son homeomorfos. El rec\'iproco no es cierto:
un c\'irculo anudado y uno desanudado en $\mathbb{R}^{3}$ son homeomorfos pero no isot\'opicos.

\section*{Funciones DL y Reconstrucci\'on}

Dado un suconjunto compacto $K\subset\mathbb{R}^{d}$ y un n\'umero real no negativo $r$, la uni\'on de
bolas de radio $r$ centradas en $K$, $K^{r} = \cup_{x\in K}B\left(x, r\right)$, llamado el
$r$-cubrimiento de $K$, es el conjunto de $r$-subnivel de la distancia
$d_{K}:\mathbb{R}^{d}\rightarrow\mathbb{R}$ definida por
$d_{K}\left(x\right) = \inf_{y\in K}\left\|x-y\right\|$; es decir,
$K^{r} = d^{-1}_{k}\left(\left[0, r\right]\right)$. Esto nos permite utilizar propiedades diferenciales
de funciones distancia y nos ayuda a comparar la topolog\'ia de los cubrimientos de conjuntos compactos
 que est\'en cercas el uno del otro con respecto a la distancia de Hausdorff.
 
\begin{definicion}
    (Distancia de Hausdorff en $\mathbb{R}^{3}$). La distancia de Hausdorff entre dos subconjuntos compactos
    $K$, $K'$ de $\mathbb{R}^{d}$ esta definida como
    \begin{equation*}
        d_{H}\left(K, K'\right) = \left\|d_{K}-d_{K'}\right\|_{\infty} =
        \inf_{x\in\mathbb{R}^{d}}\left|d_{K}\left(x\right)-d_{K'}\left(x\right)\right|
    \end{equation*}
\end{definicion}
 
Aqu\'i, los conjuntos compactos son el conjunto de datos $\mathbb{X}_{n}$ y el soporte $M$ de la
medida $\mu$. Cuando $M$ es una subvariedad compacta suave, bajo ciertas condiciones sobre
$d_{H}\left(\mathbb{X}_{n}, M\right)$, para alg\'un $r$ bien escogido, las coberturas de
$\mathbb{X}_{n}$ son homot\'opicamente equivalentes a $M$, Chazal y Lieutier (2008) \cite{Chazal2008},
Niyogi et al. (2008) \cite{Niyogi2008} (Ver Figura \ref{fig:Figura 5}). Estos resultados se extienden
a clases m\'as grandes de conjuntos compactos y llevan a resultados fuertes sobre inferencia de los
tipos de isotop\'ias de las coberturas de $M$, Chazal et al. (2009c) \cite{Chazal2009c},
Chazal et al. (2009d)\cite{Chazal2009d}. Tambi\'en llevan a resultados en la estimaci\'on de otras
cantidades geom\'etricas y diferenciales tales como normales, Chazal et al. (2009c) \cite{Chazal2009c},
curvaturas Chazal et al. (2009e) \cite{Chazal2009e}, o medidas de frontera,
Chazal et al. (2010) \cite{Chazal2010} bajo ciertas condiciones en la distancia de Hausdorff entre la
forma subyacente y los datos muestrales.
 
Estos resultados dependen de la $1$-semiconcavidad del cuadrado de la funci\'on distancia $d_{K}^{2}$,
esto es, la convexidad de la funci\'on $x\rightarrow\left\|x\right\|^{2}-d_{K}^{2}\left(x\right)$,
definida de a continuaci\'on.
 
\begin{definicion}
    Una funci\'on $\phi:\mathbb{R}^{d}\rightarrow\mathbb{R}_{+}$ es DL (distance-like) si es propia (la
    preimagen de cualquier conjunto compacto en $\mathbb{R}$ bajo $\phi$ es un compacto en
    $\mathbb{R}^{d}$) y $x\rightarrow\left\|x\right\|^{2}-\phi^{2}\left(x\right)$ es convexa.
\end{definicion}
 
Gracias a su semiconcavidad, una funci\'on DL $\phi$ tiene un gradiente
$\nabla\phi:\mathbb{R}^{d}\rightarrow\mathbb{R}^{d}$ bien definido, pero no continuo, que puede ser
integrado en un flujo continuo (Petrunin, 2007 \cite{Petrunin2007}) que permite rastrear la evoluci\'on
de la topolog\'ia de sus subniveles y compararla a una de los subniveles de funciones DL cercanas.
 
\begin{definicion}
    Sea $\phi$ una funci\'on DL y sea $\phi^{r}=\phi^{-1}\left(\left[0,r\right]\right)$ el $r$-subnivel
    de $\phi$.
    
    \begin{itemize}
        \item Un punto $x\in\mathbb{R}^{d}$ es llamado $\alpha$-cr\'itico si
        $\left\|\nabla_{x}\phi\right\|\leq\alpha$. El valor $r=\phi\left(x\right)$ correspondiente,
        tambi\'en es llamado $\alpha$-cr\'itico.
        
        \item El tama\~{n}o del atributo d\'ebil de $\phi$ en $r$ es el m\'inimo $r>0$ tal que $\phi$
        no tiene ning\'un valor cr\'itico entre $r$ y $r+r'$. Lo denotamos por
        $\mathrm{wfs}_{\phi}\left(r\right)$ (weak feature size). Para cualquier $0<\alpha<1$, el 
        $\alpha$-alcance de $\phi$ es el m\'aximo $r$ tal que $\phi^{-1}\left(\left(0,r\right]\right)$
        no contiene ning\'un punto $\alpha$-cr\'itico.
        
    \end{itemize}
\end{definicion}
 
El tama\~{n}o del atributo d\'ebil $\mathrm{wfs}_{\phi}\left(r\right)$ (respecto al $\alpha$-alcance)
mide la regularidad de $\phi$ sobre sus $r$-niveles (respecto al $O$-nivel). Cuando $\phi=d_{K}$ es
la funci\'on distancia a un conjunto compacto $K\subset\mathbb{R}^{d}$, el $1$-alcance coincide con el
alcance cl\'asico de la teor\'ia de la medida gem\'etrica, Federer (1959) \cite{Federer1959}. Su
estimaci\'on desde muestras aleatorias fue estudiada en Aamari et al. (2019) \cite{Aamari2019}. Una
propiedad importante de una funci\'on DL $\phi$ es que la topolog\'ia de sus subniveles $\phi^{r}$
s\'olo puede cambiar cuando $r$ cruza un valor $0$-cr\'itico.

\begin{lema}
    (Lema de isotop\'ia). Sea $\phi$ una funci\'on DL y $r_{1}<r_{2}$ dos n\'umeros positivos tales que
    $\phi$ no tiene puntos $0$-cr\'iticos, esto es, puntos $x$ tales que $\nabla\phi\left(x\right)=0$,
    en el subconjunto $\phi^{-1}\left(\left[r_{1},r_{2}\right]\right)$. Entonces todos los subniveles
    $\phi^{-1}\left(\left[0,r\right]\right)$ son isot\'opicos para $r\in\left[r_{1},r_{2}\right]$.
\end{lema}

\begin{figure}[ht]
    \centering
    \includegraphics[width=0.85\linewidth]{./figures/Figura5.png}
    \caption{
        Ejemplo de una nube de puntos $\mathbb{X}_{n}$ muestreada en la
        superficie de un toro en $\mathbb{R}^{3}$ y sus coberturas para
        diferentes valores del radio $r_{1}<r_{2}<r_{3}$. Para valores
        bien escogidos del radio (por ejemplo $r_{1}$ y $r_{2}$), las
        coberturas son homot\'opicamente equivalentes al toro.
    }
    \label{fig:Figura 5}
    \vspace{15pt}
\end{figure}

Como consecuencia inmediata del lema de isotop\'ia, todos los subniveles de $\phi$ entre $r$ y
$r + \mathrm{wfs}_{\phi}\left(r\right)$ tienen la misma topolog\'ia. Ahora, el siguiente teorema de
Chazal et al. \cite{Chazal2011b}, proporciona una conexi\'on entre la topolog\'ia de los subniveles
de funciones DL cercanas.

\newpage

\begin{teorema}[Teorema de reconstrucci\'on]\label{teoRecon}
    Sean $\phi$, $\psi$ dos funciones DL, tales que
    $\left\|\phi-\psi\right\|_{\infty}\leq\epsilon$, con $\alpha$-alcance
    $\mathrm{reach}_{\alpha}\left(\phi\right)\geq R$ para algunos $\epsilon$ y $\alpha$ positivos.
    Entonces, para todo $r\in\left[4\epsilon/\alpha^{2}, R-3\epsilon\right]$ y cada
    $\eta\in\left(0, R\right)$ los subniveles $\psi^{r}$ y $\phi^{\eta}$ son homot\'opicamente
    equivalentes si:
    
    \begin{equation*}
        \epsilon\leq\frac{R}{5+4/\alpha^{2}}
    \end{equation*}
\end{teorema}

Bajo condiciones similares pero ligeramente m\'as t\'ecnicas, el teorema de reconstrucci\'on puede ser
extendido para probar que los subniveles son homeomorfos e incluso isot\'opicos
(Chazal et al., 2009 \cite{Chazal2009c}; Chazal et al., 2008 \cite{Chazal2008}).

Consideremos una vez m\'as $\phi = d_{M}$ y $\psi = d_{\mathbb{X}_{n}}$ las funciones distancia
al soporte de $M$ de la medida $\mu$ y al conjunto de puntos asociados a los datos $\mathbb{X}_{n}$,
la condici\'on $\mathrm{wfs}_{\alpha}\left(d_{M}\right)\geq R$ puede ser interpretada como una
condici\'on de regularidad sobre $M$\footnote{Por ejemplo, si $M$ es una subvariedad compacta suave,
el $0$-alcance $\mathrm{reach}_{0}\left(\phi\right)$ siempre es positivo se le llama el alcance de
$M$ Federer (1959) \cite{Federer1959}}. El teorema de reconstrucci\'on junto con el teorema del nervio
nos indican que para ciertos valores de $r$, $\eta$ y los $\eta$-cobertura son homot\'opicamente
equivalentes al nervio de la uni\'on de las bolas de radio $r$ centradas en $\mathbb{X}_{n}$, es decir,
el complejo de \v Cech $Cech_{r}\left(\mathbb{X}_{n}\right)$.

Desde un punto de vista estad\'istico, la principal ventaja de estos resultados sobre la distancia de
Hausdorff es que el problema de estimaci\'on de cantidades topol\'ogicas se transforma en una serie
de preguntas acerca de el soporte de ciertas medidas, las cuales han sido ampliamente estudiadas.

\section{Inferencia Homol\'ogica}

Los resultados anteriores proporcionan una estructura matem\'atica bien fundamentada para inferir la
topolog\'ia de las formas de un complejo simplicial construido sobre una muestra finita que sirve como
aproximaci\'on. Sin embargo, desde una perspectiva m\'as pr\'actica, aparecen dos problemas. Primero, el
teorema de reconstrucci\'on requiere de regularidad a trav\'es de la condici\'on del $\alpha$-alcance
que a veces no puede ser garantizada, adem\'as de la elecci\'on del radio $r$ que se debe realizar para
construir el complejo de \v Cech $Cech_{r}\left(\mathbb{X}_{n}\right)$. Segundo,
$Cech_{r}\left(\mathbb{X}_{n}\right)$ brinda una fiel descripci\'on topol\'ogica de los datos a trav\'es
de un complejo simplicial que normalmente no es adecuado para un procesamiento de datos adicional. Es
conveniente tener descriptores topol\'ogicos que sean f\'aciles de manejar, en particular descriptores
num\'ericos, que pueden ser calculados desde el complejo simplicial de manera sencilla. Este segundo
problema se resuelve al considerar la homolg\'ia del complejo simplicial en cuesti\'on,
tema que se desarrollara a continuaci\'on, por otra parte, el primer problema sera resuelto
en la siguiente cap\'itulo con la introducci\'on a la homolog\'ia persistente.

\subsubsection*{Homolog\'ia}

La homolog\'ia es un concepto cl\'asico en la topolog\'ia algebraica, brinda una herramienta poderosa
para formalizar y manejar la noci\'on de caracter\'isticas topol\'ogicas de un espacio topol\'ogico o
un complejo simplicial de manera algebraica. Para cualquier dimensi\'on $k$, los ``hoyos''
$k$-dimensionales son representados por un espacio vectorial $H_{k}$, cuya dimeni\'on es el n\'umero
de dichas propiedades. Por ejemplo, el grupo de homolog\'ia $0$-dimensional $H_{0}$ representa las
componentes conexas del complejo, el grupo de homolog\'ia $1$-dimensional $H_{1}$ representa los lazos
de dimensi\'on uno, el grupo de homolog\'ia $2$-dimensional $H_{2}$ representa las cavidades de
de dimensi\'on dos, y as\'i sucesivamente.

Para evitar dificultades y sutilezas t\'ecnicas, restringimos esta introducci\'on a la homolog\'ia al
m\'inimo necesario para continuar con nuestro programa. En particular, nos restringimos al caso
donde la homolog\'ia tiene coeficientes en $\mathbb{Z}_{2}$, esto es, el campo con dos elementos,
$0$ y $1$, tales que $1 + 1 = 0$, que tiene una interpretaci\'on geom\'etrica m\'as intuitiva. No
obstante, todas las nociones y resultados presentados aqu\'i se extienden de manera natural a la
homolog\'ia con coeficientes en cualquier campo. Referimos al lector al estudio por Hatcher
(2001) \cite{Hatcher2001} para una introducci\'on completa a la homolog\'ia y al estudio por Ghrist
(2017) \cite{Ghrist2017} para una introducci\'on concisa y reciente a la topolog\'ia algebraica
aplicada y sus conexiones con el an\'alisis de datos.

Sea $K$ un complejo simplicial (finito) y $k$ un entero no-negativo. El espacio de las $k$-cadenas en
$K$, $C_{k}\left(K\right)$ es el conjunto cuyos elementos son las sumas formales (finitas) de los
$k$-simplices de $K$. M\'as precisamente, si $\left\{\sigma_{1},\dots,\sigma_{p}\right\}$ es el
conjunto de los $k$-simplices de $K$, entonces cualquier $k$-cadena puede ser escrita como:

\begin{equation*}
    c = \sum_{i=0}^{p}\epsilon_{i}\sigma_{i} \text{ con } \epsilon_{i}\in\mathbb{Z}_{2} 
\end{equation*}

\begin{figure}[ht]
    \centering
    \includegraphics[width=0.85\linewidth]{./figures/Figura6.png}
    \caption{
        Algunos ejemplos de cadenas, ciclos y fronteras en un complejo $K$ de dos dimensiones:
        $c_{1}$, $c_{2}$ y $c_{4}$ son $1$-ciclos; $c_{3}$ es una $1$-cadena pero no un $1$-ciclo;
        $c_{4}$ es una $1$-frontera, la frontera de la $2$-cadena obtenida de la suma de los
        tri\'angulos rodeados por $c_{4}$. Los ciclos $c_{1}$ y $c_{2}$ generan el mismo elemento en
        $H_{1}\left(K\right)$ ya que su diferencia es la $2$-cadena representada por la uni\'on de
        tri\'angulos que rodean la uni\'on de $c_{1}$ y $c_{2}$.
    }
    \label{fig:Figura 6}
    \vspace{15pt}
\end{figure}

Si $c'=\sum_{i=1}^{p}\epsilon_{i}'\sigma_{i}$ es otra $k$-cadena y $\lambda\in\mathbb{Z}_{2}$,
la suma $c+c'$ esta definida como $c+c'=\sum_{i=1}^{p}\left(\epsilon_{i}+\epsilon_{i}'\right)\sigma_{i}$
y el producto $\lambda\cdot c$ esta definido como
$\lambda\cdot c = \sum_{i=1}^{p}\left(\lambda\cdot\epsilon_{i}\right)\sigma_{i}$, convirtiendo a
$C_{k}\left(K\right)$ en un espacio vectorial con coeficientes en $\mathbb{Z}_{2}$. Ya que estamos
considerando los coeficientes en $\mathbb{Z}_{2}$, geom\'etricamente, una $k$-cadena puede ser vista como
una colecci\'on finita de $k$-simplices y la sumas de dos $k$-cadenas como la diferencia sim\'etrica
de las colecciones correspondientes\footnote{Recordemos que la diferencia sim\'etrica entre dos conjuntos
$A$ y $B$ es el conjunto $A\Delta B = \left(A\\B\right)\cup\left(B\\A\right)$}.

La frontera de un $k$-simplejo $\sigma = \left[v_{k},\dots,v_{k}\right]$ es la $\left(k-1\right)$-cadena

\begin{equation*}
    \partial_{k}\left(\sigma\right) = \sum_{i=0}^{k}\left(-1\right)
    \left[v_{0},\dots,\hat{v}_{i},\dots,v_{k}\right]
\end{equation*}

\noindent donde $\left[v_{0},\dots,\hat{v}_{i},\dots,v_{k}\right]$ es el $\left(k-1\right)$-simplejo
generado por todos los v\'ertices a excepci\'on de $v_{i}$\footnote{Ya que estamos considerando
los coeficientes en $\mathbb{Z}_{2}$, se tiene que $-1=1$ y por lo tanto $\left(-1\right)^{i}=1$
para cualquier $i$.}. Dado que los $k$-simplejos forman una base de $C_{k}\left(K\right)$,
$\partial_{k}$ se extiende como una funci\'on lineal de $C_{k}\left(K\right)$ a $C_{k-1}\left(K\right)$
llamado el operador frontera. El kernel de $\partial_{k}$ denotado por: $Z_{k}\left(K\right)=
\left\{c\in C_{k}\left(K\right):\partial_{k}\left(c\right)=0\right\}$ es llamado el espacio de
$k$-ciclos de $K$, y la imagen de $\partial_{k+1}$ denotada por: $B_{k}\left(K\right)=
\left\{c\in C_{k}\left(K\right): \exists c' \in C_{k+1}\left(K\right),
\partial_{k+1}\left(c'\right)=c\right\}$ es llamada el espacio de $k$-fronteras de $K$.

\begin{figure}[ht]
    \centering
    \includegraphics[width=0.85\linewidth]{./figures/Figura7.png}
    \caption{
        N\'umeros de Betti en el c\'irculo, la esfera de dimensi\'on dos y el toro de dimensi\'on dos.
        Las curvas azules en el toro representan dos ciclos independientes cuya clase de homolog\'ia
        es una base para el grupo de homolog\'ia de dimensi\'on uno.
    }
    \label{fig:Figura 7}
    \vspace{15pt}
\end{figure}

\newpage

El operador frontera satisface la siguiente propiedad fundamental:

\begin{equation*}
    \partial_{k+1}\circ\partial_{k+1}\equiv 0 \text{ para cualquier } k\geq 1.
\end{equation*}

\noindent en otras palabras, cualquier $k$-frontera es un $k$-ciclo, esto es,
$B_{k}\left(K\right)\subseteq Z_{k}\left(K\right)\subseteq C_{k}\left(K\right)$. Estas nociones son
ilustradas en la Figura \ref{fig:Figura 6}.

\begin{definicion}
    (grupo de homolog\'ia simplicial y n\'umeros de Betti). El $k$-\'esimo grupo de homolog\'ia
    (simplicial) de $K$ es el espacio cociente
    
    \begin{equation*}
        H_{k}\left(K\right)=Z_{k}\left(K\right)/B_{k}\left(K\right).
    \end{equation*}
    El $k$-\'esimo n\'umero de Betti de $K$ es la dimensi\'on $\beta_{k}\left(K\right)=
    \mathrm{dim}H_{k}\left(K\right)$ del espacio $H_{k}\left(K\right)$.
\end{definicion}

La Figura \ref{fig:Figura 7} muestra los n\'umeros de Betti de algunos espacios sencillos. Dos ciclos,
$c, c' \in Z_{k}\left(K\right)$, se dicen ser hom\'ologos si difieren por una frontera, esto es, si
existe una $\left(k+1\right)$-cadena $d$ tal que $c'=c+\partial_{k+1}\left(d\right)$. Dichos ciclos dan
lugar al mismo elemento de $H_{k}$. En otras palabras, los elementos de $H_{k}\left(K\right)$ son clases
de equivalencia de ciclos hom\'ologos.

Los grupos de homolog\'ia simplicial y los n\'umeros de Betti son invariantes topologicas; si $K$, $K'$
son dos complejos simpliciales tales que sus realizaciones geom\'etricas son homot\'opicamente
equivalentes, entonces sus grupos de homolog\'ia son isomorfos y sus n\'umeros de Betti son iguales.

La homolog\'ia singular es otra noci\'on de homolog\'ia que nos permite considerar una mayor variedad de
espacios topol\'ogicos. Esta definida para cualquier espacio topol\'ogico $X$ de manera similar a la
homolog\'ia simplicial, excepto que el concepto de simplejo, es reemplazado por por el de simplejo
singular, que consiste en una funci\'on continua $\sigma: \Delta_{k}\rightarrow X$ donde $\Delta_{k}$
es el simplejo est\'andar de dimensi\'on $k$. El espacio de las $k$-cadenas es el espacio
vectorial generado por los simplejos singulares $k$-dimensionales, y la frontera de un simplejo $\sigma$
esta definida como la suma (alternante) de la restricci\'on de $\sigma$ a las caras $(k-1)$-dimensionales
de $\Delta_{k}$. Algo importante acerca de la homolog\'ia singular es hecho de que esta coincide con
la homolog\'ia simplicial cuando $X$ es homeomorfo a la realizaci\'on gem\'etrica de un complejo
simplicial. Esto nos permite hablar acerca de la homolog\'ia de un espacio topol\'ogico o un complejo
simplical, sin tener que especificar si nos referimos a la homolog\'ia singular o simplicial.

Observemos que si $f:X\rightarrow Y$ es una funci\'on continua, entonces para cualquier simplejo singular
$\sigma:\Delta_{k}\rightarrow X$ en $X$, se tiene que $f\circ\sigma:\Delta_{k}\rightarrow Y$ es un
simplejo singular en $Y$, de aqu\'i, deducimos que funciones continuas entre espacios topol\'ogicos
inducen homomorfismos entre sus grupos de homolog\'ia. En particular, si $f$ es una equivalencia
homot\'opica, entonces se induce un isomorfismo entre $H_{k}\left(X\right)$ y $H_{k}\left(Y\right)$ para
cualquier $k$ entero no-negativo. Por ejemplo, sea $X\subset \mathbb{R}^{d}$ cualquier conjunto de puntos
y $r>0$, se sigue del teorema del nervio que la $r$-cobertura $X^{r}$ y el complejo de \v Cech
$Cech_{r}\left(X\right)$ tienen grupos de homolog\'ia isomorfos y los mismos n\'umeros de Betti.

%Nuevos comandos

Adem\'as de esto, tenemos como consecuencia del teorema de reconstrucci\'on \ref{teoRecon} el siguiente
resultado que nos auxilia en la estimaci\'on de n\'umeros de Betti.

\begin{teorema}
    Sea $M \subset \mathbb{R}^{d}$ un conjunto compacto con alcance,
    $\mathrm{reach}_{\alpha}\cpar{d_{M}}\geq R > 0$ para alg\'un $\alpha\in\cpar{0, 1}$ y sea $\mathbb{X}$
    un conjunto finito de puntos tales que:
    \begin{equation*}
        d_{H}\cpar{M, \mathbb{R}}=\epsilon < \frac{R}{5+4/\alpha^{2}}.
    \end{equation*}
    Entonces, para cada $r\in\ccorch{4\epsilon/\alpha^{2}, R-3\epsilon}$ y cada $\eta\in\cpar{0, R}$,
    los n\'umeros de Betti de $Cech_{r}\cpar{\mathbb{X}}$ y $M^{\eta}$ son iguales.
    
    En particular, si $M$ es una subvariedad suave de $\mathbb{R}^{d}$ de dimensi\'on $m\in\mathbb{Z}$,
    entonces $\beta_{k}\cpar{Cech_{r}\cpar{\mathbb{R}}}=\beta_{k}\cpar{M}$ para cualquier $k=0, \dots, m$.
\end{teorema}

Desde una perspectiva m\'as pragm\'atica, este resultado nos genera tres problemas: primero, la
suposici\'on de regularidad acerca del $\alpha$-alcance de $M$ puede ser demasiado restrictiva; segundo,
el c\'alculo del nervio de la uni\'on de bolas requiere de m\'etodos para probar que la uni\'on finita de
bolas sea no-vac\'ia; tercero, la estimaci\'on de n\'umeros de Betti recae en la elecci\'on del
par\'ametro $r$.

Para solucionar los problemas anteriores, Chazal y Oudot (2008) \cite{Chazaloudot2008} establecieron el
siguiente resultado que ofrece la soluci\'on a los primeros dos problemas.

\begin{teorema}
    Sea $\mathbb{M}\subseteq\mathbb{R}^{d}$ un conjunto compacto tal que $\mathrm{wfs}\cpar{M}=
    \mathrm{wfs}_{d_{M}}\cpar{0}\geq R>0$ y sea $\mathbb{X}$ un conjunto de puntos finito tal que
    $d_{H}\cpar{M,\mathbb{X}}=\epsilon <\frac{1}{9}\mathrm{wfs}\cpar{M}$. Entonces para cualquier
    $r\in \ccorch{2\epsilon, \frac{1}{4}\cpar{\mathrm{wfs}\cpar{M}-\epsilon}}$ y cualquier
    $\eta\in\cpar{0,R}$,
    \begin{equation*}
        \beta_{k}\cpar{X^{\eta}} = \mathrm{rk}\cpar{H_{k}\cpar{Rips_{r}\cpar{\mathbb{X}}}}
        \rightarrow H_{k}\cpar{Rips_{4r}\cpar{\mathbb{X}}}
    \end{equation*}
    donde $\mathrm{rk}\cpar{H_{k}\cpar{Rips_{r}\cpar{\mathbb{X}}}}
    \rightarrow H_{k}\cpar{Rips_{4r}\cpar{\mathbb{X}}}$ denota el rango de del homomorfismo inducido
    por la inclusi\'on can\'onica (continua)
    $Rips_{r}\cpar{\mathbb{X}}\hookrightarrow Rips_{4r}\cpar{\mathbb{X}}$.
\end{teorema}

Aunque este resultado deja abierta la elecci\'on del par\'ametro $r$, en el estudio realizado por
Chazal y Oudot (2008) \cite{Chazaloudot2008} se provee una descripci\'on de una estrategia multiescala
que ayuda a identificar las escalas relevantes en las cuales se puede aplicar el teorema anterior.

\newpage

\section{Aspectos estad\'isticos de la inferencia homol\'ogica}

De acuerdo a los resultados de estabilidad presentados en la secci\'on anterior, un acercamiento
estad\'istico a la inferencia topol\'ogica se relaciona fuertemente al problema de estimaci\'on de
soportes de distribuciones y estimaciones de conjuntos nivel bajo la m\'etrica de Hausdorff.
Afortunadamente se cuenta con una variedad de metodos y resultados que nos atudan a estimar el soporte de
una distribuci\'on. Por ejemplo, el estimador de Devroye y Wise
(Devroye y Wise 1980 \cite{DevroyeWise1980}) definido en una muestra $\mathbb{X}_{n}$ es tambi\'en una
cobertura  particular de $\mathbb{X}_{n}$. La tasa de convergencia de $\mathbb{X}_{n}$ y el estimador
de Devroye y Wise al soporte de la distribuci\'on para la distancia de Hausdorff fueron estudiados por
Cuevas y Rodriguez-Casal (2004) \cite{CuevasRodriguezCasal2004} en $\mathbb{R}^{d}$. Recientemente, las
tasas de convergencia minimax de estimaci\'on de variedades bajo la metrica de Hausdorff, particularmente
relevantes para la inferencia topol\'ogica, fueron estudiadas por Genovese et al. (2012)
\cite{Genovese2012}. Tambi\'en existe literatura acerca de la estimacion de los conjuntos de nivel en
varias m\'etricas (vease, por ejemplo, Cadre, 2006 \cite{Cadre2006}; Polonik, 1995 \cite{Polonik1995};
Tsybakov, 1997 \cite{Tsybakov1997}) y, particularmente, para la m\'etrica de Hausdorff Chen et al.
(2017) \cite{Chen2017}. Todos estos trabajos acerca de la estimaci\'on de soportes y conjuntos nivel,
dan lugar al an\'alisis estad\'istico de procesos de inferecia topol\'ogica.

En el estudio por Nigoyi et al (2008) \cite{Niyogi2008}, se muestra que el tipo de homotop\'ia de
variedades Riemannianas con alcance mayor que cierta constante puede ser recuperado con una
alta probabilidad de las coberturas de una muestra en (o bien, cerca) de la variedad en cuesti\'on.
Este articulo fue probablemente el primer intento de considerar la inferencia topol\'ogica en t\'erminos
de probabilidad. El estudio por Nigoyi et al.\cite{Niyogi2008} derivo de un argumento de contracci\'on de
retracto y utiliz\'o cotas estrechas sobre el n\'umero de cobertura de la variedad para controlar
la distancia de Hausdorff entre la variedad y la nube de puntos observada. La inferencia homol\'ogica
en el caso de ruido presente, esto es, en el sentido de que la distribuci\'on de la observaci\'on se
concentra alrededor de la variedad, tambi\'en fue estudiado por Nigoyi et al. (2008) \cite{Niyogi2008},
Nigoyi et al. (2011) \cite{Niyogi2008}. La suposici\'on de que el objeto geom\'etrico es una variedad
Riemanniana suave solo es usada en el art\'iculo para controlar la distancia de Hausdorff entre la
muestra y la variedad y no es realmente necesaria para la ``parte topol\'ogica'' del resultado, el cual
es similar a aquellos en los estudios por Chazal et al. (2009) \cite{Chazal2009d}, Chazal y Lieutier
(2008) \cite{Chazal2008} en el entorno particular de las variedades Riemannianas. Empezando por el
resultado del estudio por Nigoyi et al. (2008) \cite{Niyogi2008}, las tasas de convergencia minimax
del tipo de homolog\'ia han sido estudiadas por Balakrishnan et al, (2012) \cite{Balakrishnan2012}
bajo varios modelos de variedades Riemannianas con alcance m\'as grande que cierta constante. En contraste, no se ha propuesto una versi\'on estad\'istica del trabajo por Chazal et al. (2009)
\cite{Chazal2009d}.

M\'as recientemente, siguiendo las ideas encontradas en Nigoyi et al. (2008) \cite{Niyogi2008},
Bobrowski et al (2014) \cite{Bobrowski2014} se ha propuesto un robusto estimador homol\'ogico para los
conjuntos de nivel de funciones de densidad y regresi\'on, por medio de considerar la inclusi\'on entre
pares anidados de conjuntos de nivel estimados obtenidos mediante un estimador del kernel.

\section{M\'as all\'a de la Distancia de Hausdorff: Distancia a una Medida}

Es bien sabido que los m\'etodos del ATD fallan rotundamente en presencia de puntos aislados, a\~{n}adir
un solo punto asilado al conjunto de datos puede alterar la funci\'on distacia de manera dram\'atica
(ver Figura \ref{fig:Figura 8}). Como respuesta a esto, Chazal et al. (2011) \cite{Chazal2011b}
introdujeron una funci\'on distancia alternativa la cual es resistente ante el ruido, la distancia a una
medida.

Dada una distribuci\'on de probabilidad $P$ en $\mathbb{R}^{d}$ y un par\'ametro real $0\leq U\leq 1$,
la noci\'on de distancia al soporte de $P$ puede ser generalizada como la funci\'on

\begin{equation*}
    \delta_{P,u}:x\in\mathbb{R}^{d}\mapsto\inf{t>0:P\cpar{B\cpar{x,t}}\geq u}
\end{equation*}

donde $B\cpar{x,t}$ es la bola cerrada (Euclideana) con centro en $x$ y radio $t$. Para evitar problemas
de discontinuidad con la funci\'on $P\rightarrow \delta_{P,u}$, la funci\'on distancia a la medida (DAM)
con par\'ametro $m\in\ccorch{0,1}$ y potencia $r\geq 1$ esta definida como

\begin{equation}
    d_{P,m,r}\cpar{x}:
    x\in\mathbb{R}^{d}\mapsto\cpar{\frac{1}{m}\int_{0}^{m}\delta_{P,u}^{r}\cpar{x}\,du}^{\frac{1}{r}}
\end{equation}

\par Una propiedad deseable de las DAM demostrada por Chazal et al. (2011)\cite{Chazal2011b} es la
estabilidad con respecto a las perturbaciones de $P$ en la m\'etrica de Wasserstein, m\'as precisamente,
la funci\'on $P\rightarrow d_{P,m,r}$ es $m^{-\frac{1}{r}}$-Lipschitz, esto es, si $P$ y $\tilde{P}$ son
dos distribuciones de probabilidad en $\mathbb{R}^{d}$, entonces

\begin{equation}
    \cnorm{d_{P,m,r}-d_{\tilde{P},m,r}}_{\infty}\leq m^{-\frac{1}{r}}W_{r}\cpar{P,\tilde{P}}
\end{equation}

donde $W_{r}$ es la distancia de Wasserstein para la m\'etrica Euclidiana en $\mathbb{R}^{d}$, con
exponente $r$\footnote{Ver Villiani (2003)\cite{Villiani2003} para la definici\'on de distancia de
Wasserstein}. Esta propiedad implica que la DAM asociada con distribuciones cercanas en la m\'etrica de
Wasserstein tienen conjuntos subnivel cercanos. M\'as a\'un, cuando $r=2$, la funci\'on
$d_{P,m,2}^{2}$ es semiconcava, lo cual asegura fuertes propiedades de regularidad en la
geometr\'ia de sus subniveles. Usando estas propiedades, Chazal et al. (2011) \cite{Chazal2011b} mostr\'o
que bajo suposiciones generales, si $\tilde{P}$ es una distribuci\'on de probabilidad que aproxima a $P$,
as\'i los conjuntos subnivel de $d_{\tilde{P},m,2}$ proveen una aproximaci\'on topol\'ogicamente
correcta al soporte de $P$.

\par En la pr\'actica, la medida $P$ usualmente solo es conocida a trav\'es de un conjunto finito de
observaciones $\mathbb{X}_{n}=\cllav{X_{1},\dots,x_{n}}$ muestreada desde $P$, dando lugar a la pregunta
de una aproximaci\'on a la DAM. Una idea natural para estimar la DAM desde $\mathbb{X}_{n}$ es utilizar
la medida emp\'irica $P_{n}$ en lugar de $P$ en la definici\'on de la DAM. Esto corresponde al computo
de la distancia à la medida emp\'irica (DAME). Para $m=\frac{k}{n}$, la DAME satisface

\begin{equation*}
    d_{P_{n},k/n,r}^{r}\cpar{x}:=\frac{1}{k}\sum_{j=1}^{k}\cnorm{x-\mathbb{X}_{n}}_{\cpar{j}}^{r}
\end{equation*}

donde $\cnorm{x-\mathbb{X}_{n}}_{\cpar{j}}$ denota la distancia entre $x$ y su $j$-\'esima vecindad
en $\cllav{X_{1},\dots,X_{n}}$. Esta cantidad es f\'acil de calcular en la pr\'actica ya que solo
requiere de distancias entre $x$ y los puntos de la muestra. La convergencia de las DAME a las DAM
ha sido estudiada por Chazal et al. (2017)\cite{Chazal2017} y Chazal et al (2016)\cite{Chazal2016b}.

La introducci\'on de las DAM a motivado trabajos y aplicaciones en diferentes direcciones tales como
el an\'lisis topol\'ogico de datos (Buchet et al., 2015\cite{Buchet2015a}), an\'alisis de trazas GPS
(Chazal et al., 2011 \cite{Chazal2011a}), estimaci\'on de densidad (Biau et al., 2011 \cite{Biau2011}),
pruebas de hip\'otesis (Br\'echeteau, 2019 \cite{Brecheteau2019}), y agrupamiento (Chazal et al., 2013
\cite{Chazal2013b}), solo para nombrar algunos. Tambi\'en se han tomado en consideraci\'on,
aproximaciones, generalizaciones, y variantes de las DAM (Guibas et al., 2013 \cite{Guibas2013};
Phillips et al., 2014 \cite{Phillips2014}; Buchet et al., 2015 \cite{Buchet2015b};
Br\'echeteau y Levrard, 2020 \cite{Brecheteau2020}).

\begin{figure}[ht]
    \centering
    \includegraphics[width=0.85\linewidth]{./figures/Figura8.png}
    \caption{
        Efectos de los puntos aislados en los conjuntos subnivel de las funciones distancia. A\~{n}adir
        unos pocos puntos aislados a la nube puede alterar dram\'aticamente la funci\'on distancia y
        la topolog\'ia de sus coberturas.
    }
    \label{fig:Figura 8}
    \vspace{15pt}
\end{figure}
\chapter{Homolog\'ia Persistente}

La homolog\'ia persistente es una herramienta poderosa que es usada para el computo, estudio y codificaci\'on
multiescala de propiedades topol\'ogicas de familias anidadas de complejos simpliciales y espacios
topol\'ogicos. No solo provee algoritmos eficientes para calcular los n\'umeros de Betti de cada complejo
en las familias consideradas, como se requiere para la inferencia homol\'ogica cubierta en la secci\'on
anterior, sino que tambi\'en codifica la evoluci\'on de los grupos de homolog\'ia de los
complejos anidados a trav\'es de las escalas. Ideas y resultados preliminares que culminan en la
teor\'ia de la homolog\'ia persistente pueden ser encontrados desde antes del siglo XXI, en particular
en los trabajos de Barannikov (1994) \cite{Barannikov1994}, Frosini (1992) \cite{Frosini1992},
Robins (1999) \cite{Robins1999}; pero su desarrollo en su forma moderna se concreto en los trabajos
de Edelsbrunner et al. (2002) \cite{Edelsbrunner2002} y Zomorodian y Carlsson (2005)
\cite{Zomorodian2005}.

\section{Filtraciones}

Una filtraci\'on de un complejo simplicial $K$ es una familia anidada de subcomplejos
$\cpar{K_{r}}_{r\in t}$, donde $T\subseteq\mathbb{R}$, tal que para cualquier
$r, r' \in T$, si $r\leq r'$ entonces $K_{r}\subseteq K_{r'}$ y $K = \bigcup_{r\in T}K_{r}$.
El subconjunto $T$ puede ser finito o infinito. En general, una filtraci\'on de un espacio
topol\'ogico $\mathbb{M}$ es una familia anidada de subespacios $\cpar{M_{r}}_{r\in T}$,
donde $T\subseteq\mathbb{R}$, tal que para cualquier $r, r' \in T$,
si $r\leq r'$ entonces $M_{r}\subseteq M_{r'}$ y $M = \bigcup_{r\in T}M_{r}$. Por ejemplo, si
$f: \mathbb{M}\rightarrow\mathbb{R}$ es una funci\'on, entonces la familia
$M_{r} = f^{-1}\cpar{\left(-\infty, r\right]}$, $r\in\mathbb{R}$ define una filtraci\'on llamada la
filtraci\'on del conjunto subnivel de $f$.

En la pr\'actica, el par\'ametro $r\in T$ suele ser interpretado como un par\'ametro de escala, y las
filtraciones com\'unmente usadas en el ATD suelen pertenecer a uno de los siguientes dos tipos.

\section*{Filtraciones Sobre Datos}

Dado un subconjunto $\mathbb{X}$ de un espacio m\'etrico compacto $\cpar{M,\rho}$, las familias de
complejos de Vietoris-Rips $\cpar{Rips_{r}\cpar{\mathbb{X}}}_{r\in\mathbb{R}}$ y los complejos de \v Cech
$\cpar{Cech_{r}\cpar{\mathbb{X}}}_{r\in\mathbb{R}}$ son filtraciones\footnote{Aqu\'i consideramos
$Rips_{r}\cpar{\mathbb{X}} = Cech_{r}\cpar{\mathbb{X}} = \varnothing$, si $r<0$}. Aqu\'i, el par\'ametro
$r$ puede ser interpretado como la resoluci\'on con la que se considera el conjunto de datos $\mathbb{X}$.
Por ejemplo, si $\mathbb{X}$, es una nube de puntos en $\mathbb{R}^{d}$, gracias al teorema del nervio,
la filtraci\'on $\cpar{Cech_{r}\cpar{\mathbb{X}}}_{r\in\mathbb{X}}$ codifica la topolog\'ia de todo la
familia de uniones de bolas $\mathbb{X}^{r} = \cup_{x\in\mathbb{X}}B\cpar{x,r}$, cuando $0<r<\infty$.
Como la noci\'on de filtraci\'on es algo flexible, se han considerado muchas otras filtraciones en la
literatura para ser construidas sobre los datos, como el complejo testigo popularizado en el ATD por
De Silva y Carlsson (2004)\cite{DeSilva2004}, las filtraciones de Rips con peso Buchet et al.
(2015)\cite{Buchet2015b}, o las filtraciones DTM Anai et al. (2019)\cite{Anai2019} que nos permiten
trabajar con conjuntos de datos con ruido o con datos at\'ipicos.

\section*{Filtraciones de Conjuntos Subnivel}

Definir funciones en los v\'ertices de un complejo simplicial da lugar a otro importante ejemplo de
filtraci\'on: sea $K$ el complejo simplicial con el conjunto de v\'ertices $V$ y
$f:V\rightarrow\mathbb{R}$. Entonces $f$ puede ser extendida a todos los simplices de $K$ definiendo
$f\cpar{\ccorch{v_{0},\dots,v_{k}}} = \max\cllav{f\cpar{v_{i}}: i=1,\dots,k}$ para cualquier simplejo
$\sigma = \ccorch{v_{0},\dots,v_{k}}\in K$ y la familia de subcomplejos,
$K_{r}=\cllav{\sigma\in K:f\cpar{\sigma}\leq r}$, define una filtraci\'on llamada la filtraci'on del
conjunto subnivel de $f$. La filtraci\'on del conjunto sobre-nivel de $f$ se define de manera similar.

En la pr\'actica, incluso si el \'indice del conjunto es infinito, todas las filtraciones consideradas
son construidas en conjuntos finitos y son, en si, finitas. Por ejemplo, cuando $\mathbb{X}$ es finito,
el complejo de Vietoris-Rips $Rips_{r}\cpar{\mathbb{X}}$ cambia solo en un numero finito de \'indices,
$r$. Esto nos permite manejarlos de manera sencilla desde una perspectiva algebraica.

\section{Algunos Ejemplos}

Dada una filtraci\'on $\mathit{Filt} = \cpar{F_{r}}_{r\in T}$ de un complejo simplicial o un espacio
topol\'ogico, la homolog\'ia de $F_{r}$ cambia cuando $r$ incrementa; pueden aparecen nuevos componentes
conexos y algunos ya existentes pueden unirse, aros y cavidades pueden formarse o llenarse, etc. La
homolog\'ia persistente registra estos cambios, identifica las propiedades que aparecen y asocia un
tiempo de vida a cada una. La informaci\'on resultante se codifica como un conjunto de intervalos llamado
c\'odigo de barras, o bien, como un conjunto de puntos en $\mathbb{R}^{2}$ donde la coordenada de
cada punto es el punto de inicio y final de cada intervalo correspondiente.

Antes de dar una definici\'on formal, ilustraremos el concepto de homolog\'ia persistente con unos
ejemplos.

% IDE Change, change to hard-wrapping per sentence instead of hard-wrapping per max number of char.

\section*{Ejemplo 1}

Sea $f:\ccorch{0,1}\rightarrow\mathbb{R}$ la funci\'on de la Figura \ref{fig:Figura 9},
y sea $F_{r} = f^{-1}\cpar{\cpar{-\infty,r}}_{r\in\mathbb{R}}$ la filtraci\'on del conjunto subnivel de $f$.
Todos los conjuntos subnivel de $f$ son o bien vacios o la uni\'on de intervalos,
as\'i que la \'unica informaci\'on topol\'ogica no-trivial que brindan es su homolog\'ia cero dimensional, esto es, su n\'umero de componenetes conexas.
Para $r<a_{1}$, $F_{r}$ es vacio, pero para $r = a_{1}$, aparecen un primer componente conexo en $F_{a_{1}}$.
La homolog\'ia persistente registra $a_{1}$ como la ``fecha de nacimiento'' de una componente conexa la codifica como un intervalo que comienza en $a_{1}$.
Luego, $F_{r}$ permanece conexo hasta que $r$ toma el valor de $a_{2}$ donde una segunda componente conexa aparece.
La homolog\'ia persitente registra esta nueva componente conexa creando un segundo intervalo que comienza en $a_{2}$.
De manera similar, cuando $r$ alcanza el valor de $a_{3}$, una nueva componente conexa aparece y la homolog\'ia persistente crea otro intervalo comenzando en $a_{3}$.
Cuando $r$ alcanza $a_{4}$, las dos componentes creadas en $a_{1}$ y $a_{3}$ se juntan para crear una sola componente conexa.
En este paso, la homolog\'ia persistente sigue la regla de que la componente que muere es la m\'as reciente que ha aparecido en la filtraci\'on;
As\'i, el intervalo que comenz\'o en $a_{3}$ termina en $a_{4}$,
y el intervalo de persistencia que codifica el tiempo de vida de la componente nacida en $a_{3}$ es creado.
Cuando $r$ alcanza $a_{5}$, como en el caso previo la componente nacida en $a_{2}$ muere y se crea el intervalo $\cpar{a_{2},a_{5}}$.
El intervalo creado en $a_{1}$ permanece hasta el final de la filtraci\'on,
dando lugar al intervalo $\cpar{a_{1},a_{6}}$ si la filtraci\'on se detiene en $a_{6}$,
o bien, $\cpar{a_{1},\infty}$ si $r$ tiende a $+\infty$ (en cuyo caso la filtraci\'on se mantiene constante para $r>a_{6}$).
El conjunto de intervalos obtenidos que codifican el tiempo de vida de diferentes caracteristicas homol\'ogicas a lo largo de la filtrac\'on es llamado el codigo de barras de persistenca de $f$.
Cada intervalo $\cpar{a,a'}$ puede ser representado en el plano $\mathbb{R}^{2}$ por el punto $\cpar{a,a'}$.
El conjunto de puntos resultante es llamado el diagrama de persistencia de $f$.
Es de notar que la funci\'on puede tener multiples copias del mismo intervalo en su codigo de barras de persitencia.
Como consecuencia, el diagrama de persistenca de $f$ es un multiconjunto donde cada punto tiene una multiplicidad entera asociada.
Finalmente, por razones t\'ecnicas que seran claras m\'as adelante,
se a\~{n}aden al diagrama de persistencia todos los puntos de la diagonal $\Delta = \cllav{\cpar{b,d}: b=d}$ con multiplicidad infinita.

\begin{figure}[ht]
    \centering
    \includegraphics[width=0.85\linewidth]{./figures/Figura9.png}
    \caption{
        El c\'odigo de barras de persistencia y el diagrama de persistencia de la funci\'on
        $f:\ccorch{0,1}\rightarrow\mathbb{R}$.
    }
    \label{fig:Figura 9}
    \vspace{15pt}
\end{figure}

\section*{Ejemplo 2}

Sea $f: M\rightarrow \mathbb{R}$ la funci\'on eb la Figura \ref{fig:Figura 10},
donde $M$ es una superficie de dos dimensiones homeomorfa a un toro,
y sea $F_{r} = f^{-1}\cpar{\cpar{-\infty,r}}_{r\in \mathbb{R}}$ la filtraci\'on del conjunto subnivel de $f$.
La homolog\'ia persistente cero dimensional se calcula como en el ejemplo anterior,
lo cual genera las barras rojas en el codigo de barras de persistencia.
En este caso los subniveles tambi\'en almacenan informaci\'on acerca de caracter\'isticas homol\'ogicas uno dimensionales.
Cuando $r$ alcanza la altura $a_{1}$,
los conjuntos subnivel $F_{r}$ que eran homeomorfos a dos discos se vuelven homeomorfos a la uni\'on disjunta de un disco y un \'anulo,
creando un primer ciclo homologo a $\sigma_{1}$ en la Figura \ref{fig:Figura 10}.
El nacimiento de este uno-ciclo es representado por un intervalo (en azul) que comienza en $a_{1}$.
Similarmente, cuando $r$ alcanza $a_{2}$, un segundo ciclo, homologo a $\sigma_{2}$, es creado,
dando lugar al comienzo de un nuevo intervalo de persistencia.
Estos dos ciclos nunca son rellenos (abarcan $H_{1}\cpar{M}$) de manera que los intervalos que les corresponden continuan por el resto de la filtraci\'on.
Cuando $r$ alcanza $a_{3}$, un nuevo ciclo es creado, el cual se rellena en $a_{4}$,
lo cual genera el intervalo de persistencia $\cpar{a_{3},a_{4}}$.
Esta vez, la filtraci\'on del conjunto subnivel da lugar a dos codigos de barras,
uno par la homolog\'ia cero dimensional (mostrado en rojo) y otro para la homolog\'ia uno dimensional (mostrado en azul).
Estos dos codigos de barras pueden ser representados de manera equivalente como diagramas en el plano. 

\begin{figure}[ht]
    \centering
    \includegraphics[width=0.85\linewidth]{./figures/Figura10.png}
    \caption{
        El codigo de barras de persistencia y el diagrama de persistencia de la funci\'on altura
        (proyecci\'on en el eje $z$) definida en una superficie en $\mathbb{R}^{3}$.
    }
    \label{fig:Figura 10}
    \vspace{15pt}
\end{figure}

\section*{Ejemplo 3}

En este \'ultimo, consideramos la filtraci\'on dada por la uni\'on de bolas
(que crecen linealmente)
centradas en el conjunto de puntos finitos $C$ en la Figura \ref{fig:Figura 11}.
Notese que esta es la filtraci\'on del conjunto subnivel de la funci\'on distancia a $C$,
y gracias al teorema del nervio,
esta filtraci\'on es homot\'opicamente equivalente a la filtraci\'on de \v Cech construida sobre $C$.
La Figura \ref{fig:Figura 11} muestra varios conjuntos subnivel de la filtraci\'on de la siguiente manera:

\begin{enumerate}[label=\alph*)]
    \item Para radio $r=0$, la uni\'on de bolas se reduce al conjunto de puntos finito inicial,
    cada uno de ellos correspondiendo a una componente conexa;
    se comienza un intervalo por cada una de estas componentes en $r=0$.

    \item Algunas de las bolas comienzan a superponerse,
    resultando en la muerte de algunas de las componentes conexas que se han juntando entre s\'i;
    el diagrama de persistencia registra estas muertes, poniendo fin a los intervalos correspondientes.
    
    \item M\'as componentes conexas se han juntado, dejando una sola componente conexa,
    y as\'i, todos los intervalos asociados a caracter\'isticas cero dimensionales terminan,
    con la excepci\'on de los que corresponden a las componentes restantes;
    dos nuevas caracter\'isticas uno dimensionales han aparecido,
    lo cual resulta en dos nuevos intervalos (en azul) que comienzan en ese valor de $r$.
    
    \item Una de los dos ciclos uno dimensionales se ha rellenado,
    resultando en su muerte en la filtraci\'on y en el fin del intervalo correspondiente.
    
    \item Todas la componenetes uno dimensionales han muerto, dejando un \'unico intervalo rojo en el codigo de barras.
    como en ejemplos anteriores, el codigo de barras puede ser representado como un diagrama de persistencia,
    donde cada intervalo $\cpar{a,b}$ es representa por un punto en $\mathbb{R}^{2}$ de coordenadas correspondientes.
    
\end{enumerate}

Intuitivamente afirmamos que entre m\'as largo sea un intervalo en el codigo de barras,
o bien, equivalentemente,
entre m\'as alejado est\'e un punto de la diagonal en el diagrama correspondiente,
m\'as persistente, y por tanto, relevante, es la propiedad homol\'ogica que le corresponde a trav\'es de la filtraci\'on.
Es de notar tambi\'en que para un radio $r$ dado, el $k$.-\'esimo n\'umero de Betti de la uni\'on de bolas en cuesti\'on,
es igual al n\'umero de intervalos de persistencia correspondiendo a caracter\'isticas homol\'ogicas $k$ dimensionales que contienen a $r$.
As\'i, el diagrama de persistencia puede ser visto como una firma topol\'ogica que codifica la homolo\'gia de la uni\'on de bolas abiertas,
para todos los radios, as\'i como su evoluci\'on a trav\'es de los valores que toma $r$.

\begin{figure}[ht]
    \centering
    \includegraphics[width=0.85\linewidth]{./figures/Figura11.png}
    % \caption{
    %     Placeholder
    % }
    \label{fig:Figura 11}
    \vspace{15pt}
\end{figure}

\newpage

\section{Modulos y Diagramas de Persistencia}

Inicia nueva secci\'on.



%%%%%%%%%%%%%  APENDICES  %%%%%%%%%%%%%
\part*{\addcontentsline{toc}{part}{Ap\'endices}{Ap\'endices}}
\fancyhead[RO]{\leftmark}
\fancyhead[EL]{\textbf{Ap\'endice \thechapter}}
\appendix
\appendix

\chapter{Cosas que no deber\'ian ir en el texto principal}\label{Codigos}

Un ap\'endice, por qu\'e no?!

%%%%%%%%%%%%%
\backmatter
%%%%%%%%%%%%%

% Adjustments headers
\fancyhead[RO]{\leftmark}
\fancyhead[EL]{}
\addcontentsline{toc}{chapter}{Bibliograf\'ia}
\bibliographystyle{abbrv}
\bibliography{BibliografiaTesis}

\end{document}