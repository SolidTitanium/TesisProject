\addcontentsline{toc}{chapter}{Introducci\'on}
\chapter*{Introducci\'on}
%\fancyhf{}
%\rhead{INTRODUCCI\'ON}

El an\'alisis topol\'ogico de datos (ATD) es un campo reciente 
que emerge de varios trabajos en topolog\'ia (algebraica) aplicada
y la geometr\'ia computacional durante la primera d\'ecada del siglo \textbf{XXI}. Aunque
es posible encontrarse con acercamientos geom\'etricos al
an\'alisis de datos desde mucho antes, el ATD comenz\'o a desarrollarse
como un campo con los trabajos de Edelsbrunner et al. (2002) \cite{Edelsbrunner2002} y
Zomorodian y Carlsson (2005) \cite{Zomorodian2005} en homolog\'ia persistente, el campo fue
popularizado en un destacado art\'iculo en 2009 \cite{Carlsson2005}.
El ATD es motivado principalmente por la idea que la topolog\'ia y la
geometr\'ia brindan un acercamiento poderoso para inferir de manera robusta
caracter\'isticas cualitativas y cuantitativas sobre la estructura
de un conjunto de datos [e.g., Chazal (2017) \cite{Chazal2017}].

El objetivo del ATD es generar m\'etodos matem\'aticos, estad\'isticos y algor\'itmicos
bien fundamentados para inferir, analizar y explotar las complejas estructuras topol\'ogicas
y geom\'etricas subyacentes a datos que usualmente son representados como nubes de puntos en
espacios Euclideanos o espacios m\'etricos m\'as generales. En el transcurso de los \'ultimos
a\~{n}os se ha realizado un esfuerzo considerable para proporcionar estructuras de datos robustas
y eficientes, adem\'as de algoritmos para ATD que actualmente son implementados y
facilitados a trav\'es de paqueter\'ias est\'andar como la paqueter\'ia
GUDHI\footnote{https://gudhi.inria.fr/} (C++ y Python) Maria et al. (2014) \cite{Maria2014} y su
interfaz para el software R, Fasy et al. (2014a) \cite{Fasy2014a},
Dionysus\footnote{https://www.mrzv.org/software/dionysus/},
PHAT\footnote{https://bitbucket.org/phat-code/phat},
DIPHA\footnote{https://github.com/DIPHA/dipha} o
Giotto\footnote{https://giotto-ai.github.io/gtda-docs/0.4.0/library.html}.
Aunque evoluciona con rapidez, el ATD proporciona un conjunto de herramientas maduras y
eficientes que pueden ser usadas de manera complementaria o conjunta a otras herramientas
de la ciencia de datos.

\section*{Estructura General del An\'alisis Topol\'ogico de Datos}

El ATD se ha desarrollado recientemente en m\'ultiples direcciones y campos de aplicaci\'on.
Actualmente existe una variedad de m\'etodos inspirados por acercamientos topol\'ogicos y
geom\'etricos. Dar un resumen que cubra con entereza de los acercamientos
existentes se encuentra fuera del alcance de esta introducci\'on. Sin embargo, muchos m\'etodos
est\'andar siguen la siguiente secuencia:

\begin{enumerate}
    \item Suponemos que la entrada de datos es un conjunto finito de puntos con una noci\'on
    de distancia o similitud entre ellos. Esta puede ser inducida por una m\'etrica en el
    espacio de entrada (e.g. la m\'etrica Euclidiana si se trata de datos inmersos en
    $\mathbb{R}^{d}$) o ser una m\'etrica intr\'inseca definida por una matriz de distancia
    por pares. La definici\'on de la m\'etrica en los datos normalmente es parte de la entrada
    o es guiada por la aplicaci\'on. No obstante, es importante notar que la elecci\'on de
    dicha m\'etrica puede ser cr\'itica para revelar caracter\'isticas topol\'ogicas y
    geom\'etricas interesantes de los datos.
    
    \item Se construye una figura ``continua'' sobre los datos con el prop\'osito de resaltar
    las estructuras topol\'ogicas y geom\'etricas subyacentes. Usualmente se trata de
    un complejo simplicial o una familia anidada de complejos simpliciales, llamada filtraci\'on,
    la cual refleja la estructura de los datos en diferentes escalas. Los complejos simpliciales
    pueden ser vistos como generalizaciones de gr\'aficas vecinales que cl\'asicamente son
    construidas sobre los datos en muchos tipos de an\'alisis o algoritmos de aprendizaje.
    El desaf\'io aqu\'i es definir tales estructuras de tal manera que sean capaces de
    reflejar informaci\'on relevante acerca de la estructura de los datos y que puedan ser
    construidas de manera efectiva y manipuladas en la pr\'actica.
    
    \item Informaci\'on topol\'ogica y geom\'etrica es extra\'ida de las estructuras construidas
    sobre los datos. Esto puede resultar en una reconstrucci\'on completa, t\'ipicamente una
    triangulaci\'on, de la forma subyacente de los datos de los cuales se pueden extraer
    f\'acilmente propiedades topol\'ogicas y geom\'etricas en forma de res\'umenes o
    aproximaciones las cuales requieren m\'etodos espec\'ificos, como la homolog\'ia persistente,
    para la extracci\'on de informaci\'on relevante. M\'as all\'a de la identificaci\'on
    de informaci\'on topol\'ogica/geom\'etrica interesante y su visualizaci\'on e
    interpretaci\'on, el desaf\'io en este paso es mostrar su relevancia, en particular su
    estabilidad con respecto a las perturbaciones o la presencia de ruido en los datos de entrada.
    Es por ello que entender el comportamiento estad\'istico de las propiedades inferidas es
    tambi\'en una cuesti\'on importante.
    
    \item La informaci\'on topol\'ogica y geom\'etrica proporciona una nueva familia de
    caracter\'isticos y descriptores de los datos. Estos pueden ser usados para entender mejor
    los datos (en particular a trav\'es de visualizaci\'on) o pueden ser combinados con otros
    tipos de caracter\'isticos para un an\'alisis posterior o tareas de aprendizaje autom\'atico.
    Esta informaci\'on tambi\'en puede ser utilizada para dise\~{n}ar modelos bien ajustados para el
    an\'alisis de datos o el aprendizaje autom\'atico. Mostrar el valor a\~{n}adido y complementario
    (con respecto a otras caracter\'isticas) de la informaci\'on proporcionada por las
    herramientas del ATD es un punto importante en este paso.
    
\end{enumerate}

\section*{El An\'alisis Topol\'ogico de Datos y la Estad\'istica}

Hasta hace poco, los aspectos te\'oricos del TDA y la inferencia topol\'ogica reca\'ian
principalmente en acercamientos determin\'isticos. Estos acercamientos no tomaban en cuenta la
naturaleza aleatoria de los datos y la variabilidad intr\'inseca de las cantidades topol\'ogicas
que infieren. As\'i, la mayor\'ia de los m\'etodos correspondientes son de car\'acter explicativo,
sin ser capaces de distinguir eficientemente entre informaci\'on y lo que normalmente es llamado
``ruido topol\'ogico''.

Un acercamiento estad\'istico al ATD implica considerar los datos como generados de una
distribuci\'on desconocida y a su vez que las propiedades topol\'ogicas inferidas utilizando
m\'etodos del ATD son vistos como estimadores de cantidades topol\'ogicas que describen un
objeto subyacente. Bajo este acercamiento, el objeto desconocido usualmente corresponde al soporte
de la distribuci\'on de los datos (o parte del mismo). Los objetivos principales de un acercamiento
estad\'istico al an\'alisis topol\'ogico de datos pueden ser abreviados como la siguiente lista
de problemas:

\begin{enumerate}[label=\emph{T\'opico \arabic*:}]
    \item Demostrar consistencia y estudiar la tasa de convergencia de los m\'etodos del ATD.
    
    \item Proporcionar regiones de confianza para caracter\'isticas topol\'ogicas y discutir la
    significacia de las cantidades topol\'ogicas estimadas.
    
    \item Seleccionar escalas relevantes en las cuales el fen\'omeno topol\'ogico debe ser considerado,
    en funci\'on de los datos observados.
    
    \item Lidiar con valores at\'ipicos y brindar m\'etodos robustos para el ATD.
    
\end{enumerate}

\section*{Aplicaciones del An\'alisis Topol\'ogico de Datos en la Ciencia de Datos.}

Desde el punto de vista de las aplicaciones, recientemente hay muchos resultados prometedores que han
demostrado la eficacia de acercamientos topol\'ogicos y geom\'etricos en una multitud de campos, tales
como la ciencia de materiales (Kramar et al., 2013 \cite{Kramar2013}; Nakamura
et al., 2015 \cite{Nakamura2015}; Pike et al., 2020 \cite{Pike2020}), an\'alisis de
formas 3D (Skraba et al., 2010 \cite{Skraba2010}; Turner et al., 2014b \cite{Turner2014b}), an\'alisis
de im\'agenes (Qaiser et al., 2019 \cite{Qaiser2019}; Rieck et al., 2020 \cite{Rieck2020}), an\'alisis
de series de tiempo multivariadas (Khasawneh y Munch, 2016 \cite{Khasawneh2016};
Seversky et al., 2016 \cite{Seversky2016}; Umeda, 2017 \cite{Umeda2017}),
medicina (Dindin et al., 2020 \cite{Dindin2020}), biolog\'ia (Yao et al., 2009 \cite{Yao2009}),
gen\'omica (Carri\`ere y Rabad\'an, 2020 \cite{Carriere2020}), qu\'imica (Lee et al.,
\cite{Lee2017}; Smith et al., 2021 \cite{Smith2021}), redes sensoriales (De Silva y Ghrist, 2007
\cite{Silva2007}) y transportaci\'on (Li et al., 2019 \cite{Li2019}), entre otros. Dar una lista
exhaustiva de las aplicaciones del ATD esta fuera del alcance de esta introducci\'on. Por otra parte,
la mayor\'ia de los resultados del ATD son fruto de su combinaci\'on con otras t\'ecnicas de an\'alisis
y aprendizaje. De esta manera vemos que clarificar la posici\'on y complementariedad del ATD con respecto
a otros acercamientos y herramientas en la ciencia de datos es una cuesti\'on importante y un campo de
investigaci\'on activo.

As\'i, los objetivos generales de este documento son los siguientes. Primero, se intenta proporcionar
a los analistas de datos con una breve pero exhaustiva introducci\'on a los fundamentos matem\'aticos
y estad\'isticos del ATD. Con este prop\'osito, nos enfocamos en una selecci\'on de herramientas
y t\'opicos, los complejos simplicales y su uso para el an\'alisis topol\'ogico de datos exploratorio,
la inferencia geom\'etrica  y la homolog\'ia persistente, los cuales juegan un rol central en el ATD.
Segundo, se apunta a demostrar como, gracias al reciente progreso del software, herramientas del ATD
pueden ser f\'acilmente aplicadas en la ciencia de datos. En particular, mostraremos como la versi\'on
de Python de la paqueter\'ia GUDHI permite una sencilla implementaci\'on y uso de las herramientas
presentadas. Nuestro objetivo es proporcionar al analista de datos referencias relevantes de manera que
se obtenga un comprensi\'on clara de los elementos b\'asicos del ATD y sea capaz de utilizar sus
m\'etodos y software en un conjunto propio de problemas y datos.

Otros estudios del ATD, complementarios a este trabajo, pueden ser encontrados en la literatura.
Wasserman (2018) \cite{Wasserman2018} presenta una perspectiva estad\'istica al ATD, y se concentra, en
particular, en las conexiones entre el ATD y el agrupamiento por densidad. Sizemore et al. (2019)
\cite{Sizemore2019} propuso un estudio acerca de las aplicaciones del ATD a las neurociencias.
Finalmente, Hensel et al. (2021) \cite{Hensel2021} presenta un resumen de las aplicaciones
del ATD al aprendizaje autom\'atico.